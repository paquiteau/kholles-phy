\begin{Exercise}[title=Guitare]
	Les frettes placées le long du manche d’une guitare permettent au musicien de modifier la hauteur du son produit par la corde. En pressant la corde contre une frette, il diminue sa longueur, provoquant une augmentation de la fréquence fondamentale de vibration de la corde.
	\Question Retrouver rapidement la fréquence de vibration fondamentale d’une corde de longueur L le long de laquelle les ondes se propagent à la célérité v.
	\Question La note monte d’un demi-ton lorsque la fréquence est multipliée par $2^{1/12}$. Pour cela, comment doit-on modifier la longueur de la corde ?
	\Question En plaçant le doigt sur les frettes successives on monte à chaque fois la note d’un demi-ton. Combien de frettes peut-il y avoir au maximum, sachant que la distance d entre la dernière frette et le point d’accrochage de la corde (voir figure ci-dessus) doit être supérieure à 0,25L?
\end{Exercise}
\begin{Answer}
	\Question Dans le mode fondamental , on a seulement deux nœud de vibration aux extrémités : $L=\frac{\lambda}{2}$ donc $f_1=\frac{c}{2L}$
	\Question il faut multiplier sa longueur par $2^{-1/12}$
	\Question En appuyant sur la $p^{eme}$ frette la note est montée de p demi-tons. d'ou :
	\[ L_p=2^{-p/12}L \geq \frac{L}{4} ~\text{ soit }~ 2^{-p/12} \geq \frac{1}{4} ~\text{ soit }~ p\leq 24\]
\end{Answer}
