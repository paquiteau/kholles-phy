\begin{Exercise}[title=Onde sonore stationnaires]
  On dispose de deux haut-parleurs, alimentés en parallèle par un générateur
  basses fréquences (GBF) de fréquence $f = 1000 Hz$ , ainsi que de deux
  microphones. La vitesse du son est de \SI{340}{\m\per\s} . On utilise les deux
  haut-parleurs, placés face à face à une distance d, aux points $O$ et $A$ de l’axe
  $Ox$. On note $e(t) = e_0 cos (2\pi ft)$ la tension délivrée par le GBF. On supposera
  que la présence d’un haut-parleur ne perturbe pas l’onde produite par l’autre
  haut-parleur, et notamment n’engendre pas d’onde réfléchie. Chaque
  haut-parleur est supposé émettre une onde acoustique de même phase que la
  tension d’alimentation et on négligera toute atténuation des ondes sonores.
  \Question Donner la forme générale de l’onde engendrée par le haut-parleur de
  gauche $p_g(x, t)$
  \Question Exprimer l’onde engendrée par le haut-parleur de droite $p_d(x, t)$.
  On fera particulièrement attention au fait qu’en $x = d$ l’onde $p_d$ doit
  posséder la même phase que la tension d’alimentation.

  \Question L’onde entre les deux haut-parleurs est la superposition des deux
  ondes déterminées ci-dessus. On désire qu’au niveau du haut-parleur de gauche
  se forme un nœud de vibration. Exprimer les distances $d_n$ que l’on peut
  alors choisir en fonction de la longueur d’onde $\lambda$ et d’un entier $n$.

  \Question Qu’en est-il au niveau du haut-parleur de droite ? Tracer l’allure des
  ondes obtenues à un certain temps pour les trois entiers les plus faibles.

  \Question Reprendre les deux dernières questions si on veut obtenir sur le
  haut- parleur de gauche un ventre de vibration. Expliquer qualitativement la
  condition obtenue.
\end{Exercise}
\begin{Answer}
  \Question
  Le haut-parleur en $O$ émet une onde progressive sinusoïdale se
propageant selon les $x$ croissants, soit $p_g (x, t) = A\cos(\omega t - kx)$ où $A$
est l’amplitude de l’onde, $\omega = 2\pi f$ et $k = c$ où c’est la vitesse de
l’onde sonore. On vérifie qu’en effet en $x = 0$ la phase de l’onde est la
même que la tension d’alimentation, ie $\omega t = 2\pi ft$.
  \Question Le haut-parleur situé en $A$ émet une onde progressive sinusoïdale se
propageant selon les $x$ décroissants, soit  $p d (x, t) = A cos(\omega t + kx +
\phi)$ où $\phi$ est une constante. En $x = d$ la phase doit être égale à $\omega t =
\omega t + kd + \phi$ donc $\phi = -kd$ . Donc it $p_d(x, t) = A\cos(\omega t + k(x -
d))$

\Question $p(x,t) = p_d(x, t) + p_g(x, t) = A\cos(\omega t - kx) + A\cos(\omega t + k(x - d)) = 2Acos (\omega t-\frac{kd}{2})\cos(kx - \frac{kd}{2})$
Pour qu’un nœud de vibration se forme au niveau du haut-parleur de  gauche, il faut que $p(0, t) = 0$ à chaque instant soit$ cos (\frac{kd}{2}) = 0$ où $\frac{kd}{2}= \frac{\pi}{2}+n\pi$ donc $d_n=\frac{\lambda}{2}+n\lambda$
\Question Au niveau du haut-parleur de droite, la pression est exactement la
même : c’est aussi un nœud de vibration.
\Question Cette fois-ci il faut que l’amplitude de vibration en x = 0 soit maximale, soit $\cos(\frac{kd}{2})=\pm 1$. On en déduit $\frac{kd}{2} = n\pi$ ou encore $d_n'=n\lambda$. L’autre haut-parleur est aussi un ventre de vibration. Si les deux haut-parleurs sont séparés d’une distance égale à un nombre entier de longueurs d’onde, l’onde partant d’un haut-parleur garde la même phase
en arrivant sur l’autre haut-parleur. Les interférences y sont alors
constructives.



\end{Answer}
