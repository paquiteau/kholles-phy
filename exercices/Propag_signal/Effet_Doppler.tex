\begin{Exercise}[title=Effet Doppler]
	Une onde sinusoidale de fréquence $f$ se propage dans la direction de $(Ox)$ dans le sens positif à une célérité $c$. Un observateur se déplace avec une vitesse $\vec{v}=v\vec{u_x}$ parallèle à $(Ox)$.
	\Question Écrire le signal $s(x,t)$ de l'onde en définissant les notations nécessaire.
	\Question Pour l'observateur en mouvement Le point d'abscisse $x$ est repéré par une abscisse le long d'un axe $(Ox')$ qui lui est lié telle que $x'=x-vt$. Exprimer $s(x',t)$.
	\Question En déduire la fréquence $f'$ perçue par l'observateur en mouvement. Comparer $f'$ et $f$ suivant le signe de $v$
	\Question Dans la rue un camion de pompier vous dépasse , sirène en marche . Qu'entendez vous ?
\end{Exercise}
\begin{Answer}
	\Question $s(x,t)= A\cos(2\pi f (t-\frac{x}{c})+\phi)$
	\Question $s(x,t)= A\cos(2\pi f (1-\frac{v}{c})-2\pi f \frac{x'}{c}+\phi)$ \\
	\Question On a donc : $f' = f(1-\frac{v}{c})$
	\Question son grave puis aigue .
\end{Answer}
