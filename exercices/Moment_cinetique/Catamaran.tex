\begin{Exercise}[title=Catamaran]
	SCHEMA
	On considère le modèle d'un catamaran représenter sur la figure suivante:
	Cet exercice s'intéresse au condition d'équilibre du catamaran sur l'eau.
	\Question Exprimer la variation de la poussée d'archimède quand un réservoire s'enfonce d'un niveau $\epsilon$.
	\Question on considère la situation suivante:
	\subQuestion Décrire qualitativement l'évolution du système
	\subQuestion À quelle condition sur la position du centre de gravité le catamaran retrouve-t-il l'équilibre ?
\end{Exercise}
\begin{Answer}
	\Question On a $\vec{\Pi} = -\rho_{eau}V_{im}\vec{g}$. En s'enfonçant de $\epsilon$ on a une variation de volume $\Delta V = 2RL \epsilon $ donc $\Delta \vec{\Pi} = -\rho_{eau}2RL\epsilon\vec{g}$.
	\Question $\vec{Pi}= -\frac{m}{2}g -\rho_{eau}2RL\epsilon \vec{g}$
	On a  $\epsilon_1 = \xi +a \theta$ et $\epsilon_2 = \xi -a\theta$.
	Dans une situation de quasi équilibre on a:
	\[\vec{\Pi}_1+\vec{\Pi}_2+\vec{P} = \vec{0} \implies \xi = 0 \]
	On applique le TMC en G :
	$\vec{\mathcal{M}}(\vec{\Pi}_{A1}) + \vec{\mathcal{M}}(\vec{\Pi}_{A1}) = J \ddot{\theta} \vec{z}$
	on a $\mathcal{M}_{G}(\vec{\Pi_1}) = (h\theta-a)\left(\frac{mg}{2}+2\rho g  R L a\theta \right)$
	Ainsi
	\[
	\theta (\frac{mgh}{2} -4\rho_{eau} RL ga^2) = J \vec{\theta}
	\]

	Stable si $ h\leq \frac{4\rho RLa^2}{m}$
\end{Answer}
