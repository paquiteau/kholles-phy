\begin{Exercise}[title=(*) Diffusion de Rutherford]
	L'expérience consiste à bombarder une mince feuille d'or avec des particules
    $\alpha$ (noyau d'hélium $He^{2+}$). Une particule $\alpha$ de masse $m \simeq 4m_p$ arrive
    avec une vitesse $v_0$ dont le support est distant de $b$ du noyau d'or
    $^{197}_{79}Au$.

    \begin{center}
      \begin{tikzpicture}
        \draw (0,0) -- (10,0);
        \draw[thick,-latex] (0.2,0) -- ++(1,0) node[midway,above]{$\overrightarrow{v_0}$};
        \draw[latex-latex] (8,0) -- ++(0,-2) ;
        \node (F) at (8.5,-2) {$\bullet$};
        \node[left] at (F) {$F$};
      \end{tikzpicture}
    \end{center}

    \Question Quelle la force qui s'exerce sur la particule ? pourquoi peut
    on considérer le noyau d'or immobile pendant l'interaction?
    \Question Pourquoi l'énergie et le moment cinétique de la particule $\alpha$ sont
    il constant, les calculer. ?
    \Question Dans un premier temps $b=0$ et le mouvement de la particule est
    rectiligne. Quelle est la distance $r_m$ minimale d'approche de la particule
    au noyau d'or?
    \emph{Donnée : \gdr{m_p}{1,67e-27}{\kg},\gdr{e}{1,6e-19}{C}}
	si on veux $r_m$ de l'ordre de la dimension du noyau ($10^{-15}$) quelle
    doit etre la vitesse initiale?
    \Question Dans le cas $b\neq 0$ montrer que l'énergie s'écrit $E=
    \frac{1}{2}m\dot{r}+V_{eff}(r)$. Étudier et tracer $V_{eff}(r)$. Quand
    $r=r_m$ que vaut $\dot{r}$ ? Exprimer $r_m$ en fonction des données du
    problème. Étudier et interpréter les variations de $r_m$ avec E et b.
\end{Exercise}
\begin{Answer}
  \Question Le noyaux d'or et la particule $\alpha$ portent tous les deux une charges positive: la force coulombienne est donc répulsive:
  \[
    \vec{f} = \frac{2eZe}{4\pi\epsilon_0r^2}\vec{u_r}, \text{ soit } k = \frac{2Ze^2}{4\pi\epsilon_0}
  \]
  Le noyau d'or est 50 fois plus massif que la particule, on le considère fixe.
  \Question La force conservative dérive de l'énergie potentielle donc:
  \[
    E= \frac{1}{2}mv^2+\frac{k}{r} = Cste = \frac{1}{2}mv_0^2
  \]
  La force est centrale : le moemnt cinétique en $F$ est constant. Avec les conditions initiales: $\vec{\sigma}= mv_0 b \vec{u_z}$
  \Question En $r_m$, $v=0$ soit :
  \[
    r_m = \frac{Ze^2}{4\pi\epsilon_0m_pv_0^2}
  \]
  A.N. $v_0 =\left(frac{Ze^2}{4\pi\epsilon_0m_pr_m}\right)^{1/2} =1,0.10^8 m/s$. tres grande énergie !
  \Question
  \[
    E = \frac{1}{2}m\dot{r} + \frac{\sigma^2}{2mr^2} +E_p(r) 
  \]
  En $r_m$ on a $\dot{r}=0$ soit:
  \[
    E = V_{eff} = \frac{mv_0^2b^2}{2r^2}+\frac{k}{r} \implies r^2-\frac{k}{E}r-b^2 = 0
  \]
  On a la solution:
  \[
    r_m = \frac{k}{2E}+\sqrt{\frac{k^2}{4E^2}+b^2}
  \]
  On retrouve le résultat précédent pour $b=0$. si $b$ est grand , la particule n'est pas dévié ($r_m=b$).
\end{Answer}
