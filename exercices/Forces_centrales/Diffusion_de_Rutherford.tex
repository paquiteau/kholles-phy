\begin{Exercise}[title=(*) Diffusion de Rutherford]
	L'expérience consiste à bombarder une mince feuille d'or avec des particules
    $\alpha$ (noyau d'hélium $He^{2+}$). Une particule $\alpha$ de masse $m \simeq 4m_p$ arrive
    avec une vitesse $v_0$ dont le support est distant de $b$ du noyau d'or
    $^{197}_{79}Au$.
    \Question Quelle la force qui s'exerce sur la particule ? pourquoi peut
    on considérer le noyau d'or immobile pendant l'interaction?
    \Question Pourquoi l'énergie et le moment cinétique de la particule $\alpha$ sont
    il constant, les calculer. ?
    \Question Dans un premier temps $b=0$ et le mouvement de la particule est
    rectiligne. Quelle est la distance $r_m$ minimale d'approche de la particule
    au noyau d'or?
    \emph{Donnée : \gdr{m_p}{1,67e-27}{\kg},\gdr{e}{1,6e-19}{C}}
	si on veux $r_m$ de l'ordre de la dimension du noyau ($10^{-15}$) quelle
    doit etre la vitesse initiale?
    \Question Dans le cas $b\neq 0$ montrer que l'énergie s'écrit $E=
    \frac{1}{2}m\dot{r}+V_{eff}(r)$. Étudier et tracer $V_{eff}(r)$. Quand
    $r=r_m$ que vaut $\dot{r}$ ? Exprimer $r_m$ en fonction des données du
    problème. Étudier et interpréter les variations de $r_m$ avec E et b.
\end{Exercise}
\begin{Answer}
	~
\end{Answer}
