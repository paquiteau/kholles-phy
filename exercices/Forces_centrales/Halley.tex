\begin{Exercise}[title=La comète de Halley]
  La comète de \textsc{Halley} décrit une ellipse dont un des foyers est le centre du
  Soleil que l’on suppose immobile (on travaille dans le référentiel
  héliocentrique). L’aphélie A (point le plus éloigné du Soleil) se trouve à une
  distance \gdr{r_A}{5.30e9}{km} et la vitesse de la comète vaut alors
  \gdr{v_A}{0.9}{\km\per\s}
  \emph{Données: \gdr{\mathcal{G}}{6.67e-11}{USI}, masse du soleil
    \gdr{M_S}{2e30}{kg}}
  \Question On cherche à calculer les caractéristiques$r_P$ et $v_P$ de la
  trajectoire au périhélie $P$ (point le plus proche du Soleil)
  \subQuestion Trouver deux relations liant $r_A,v_A,r_P$ et$v_P$.
  \subQuestion  En déduire les valeur numérique de $r_P$ et $V_P$
  \Question Calculer la valeur du demi-grand axe $a$ de l'ellipse parcourue par
  la comète de \textsc{Halley}
  \Question Déterminer la période de la comète de \textsc{Halley}.
\end{Exercise}
\begin{Answer}
  \Question \subQuestion $v_Ar_A=v_Pr_p$ et
  $\frac{1}{2}v_A^2-\frac{\mathcal{G}M_S}{r_A}
  =\frac{1}{2}v_P^2-\frac{\mathcal{G}M_S}{r_P}$
  \subQuestion \gdr{v_P}{55}{\km\per\s} et \gdr{r_P}{8.7e7}{km}
  \Question \gdr{a}{2.7e9}{km}
  \Question $T_c=T_T \left(\frac{a}{r_T}\right)^{3/2}=76~ ans$
\end{Answer}
