\begin{Exercise}[title=(*) Limite de Roche]
On cherche à déterminer la distance en dessous de laquelle une comète s'approchant de Jupiter se sépare en plusieurs morceaux sous l'effet des forces de marées du à jupiter. Pour cela on fait les hypothèse suivantes:
\begin{itemize}
	\item Le réferentiel Jupitérocentrique est galiléen et Jupiter est un astre sphérique homogène.
	\item La comète de masse volumique $\mu_c$ est en orbite circulaire de rayon $d$ autour de jupiter.
	\item La comète est constituée de deux sphères identique de masse $m$ et de rayon$r$ homogènes elles sont liées entre elle par leur attraction gravitationnelle mutuelle.
\end{itemize}
\Question Établir que le mouvement du centre d'inertie de la comète est uniforme puis déterminer l'expression de $\omega^2$ carré de la vitesse angulaire du mouvement.
\Question On note $R$ la réaction de la sphère  de la comète la plus proche de jupiter sur la sphère la plus éloignée. À quelle condition le contact entre les deux sphères est-il rompu , à quelle distance $d_lim$ cela arrive t-il?
\Question faire l'application numérique de $\frac{d_lim}{R_J}$.
\emph{Données : \gdr{M_j}{1,9e27}{kg}; \gdr{R_j}{7,1e4}{km}; \gdr{\mu_c}{1e3}{\kg\per\m\cubed}}
\end{Exercise}
\begin{Answer}
De l'art de bien faire un DL.
\end{Answer}
