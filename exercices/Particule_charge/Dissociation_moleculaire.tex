\begin{Exercise}[title=Dissociation moléculaire]
  On s'intéresse à la rupture d'une molécule diatomique sous l'effet d'un champ
  électrique. On modélise la molécule par un couple de masses m reliées par un
  ressort de constante de raideur k et on note
  $\omega_0=\sqrt{\frac{k}{m}}$
  On considère que l'atome 1 reste immobile et que seul l'atome 2 se déplace. On
  note $\vec{r}=M_1M_2$ . On considère que la molécule est rompue si $r > R$ . Pour
  modéliser la polarisation de la liaison, on supposera que l'atome 2 porte une
  charge électrique $q$ . On se placera dans la suite du problème en régime
  forcé stationnaire établi.
  \Question La molécule est soumise à champ électrique
  $\vec{E}=E_0\cos{\omega_0t}\vec{e_x}$
  \subQuestion Montrez en quoi une solution de la forme particulière de la forme
  $\vec{r}=\vec{r_0}\cos({\omega_0t+\phi})$ ne peux pas convenir
  \subQuestion On cherche la solution sous la forme
  $\vec{r}=(A\cos(\omega_0t)+B\sin(\omega_0t))t \vec{e_x}$ Déterminer $A$ et $B$
  \subQuestion Quelle valeur minimale doit avoir le champ $\vec{E}$ pour rompre
  la molécule ?
  \Question On modifie à présent le modèle de la molécule pour faire apparraitre
  un facteur de qualité Q.
  \subQuestion Déterminez l'espression de $\vec{r}$
  \subQuestion Quelle valeur minimale doit avoir le champ $\vec{E}$ pour rompre
  la molécule?
\end{Exercise}
\begin{Answer}
  En régime stationnaire, on suppose que la composante issue de l'équation
  homogène a disparu et on ne considère plus que la contribution de la solution
  particulière. Par conséquent, on ne tiendra aucun compte des conditions
  initiales, disparues avec le régime transitoire
  \Question Cas parfait : (par analyse synthèse)
  \subQuestion le PFD donne:
  $\ddd{\vec{r}}{t} +\omega_0^2 \vec{r} = \frac{qE_0}{m}\cos(\omega_0t)\vec{e_x}$
  On injecte la solution proposée , elle n'est valable que pour $E_0=0$
  \subQuestion on a cette fois:
  \[
 \dd{\vec{r_p}}{t}=(A\cos(\omega_0t)+B\sin(\omega_0t)-A\omega_0t\sin(\omega_0t)+B\omega_0t\cos(\omega_0t))\vec{e_x}
  \]
  \[\ddd{\vec{r_p}}{t} =(-2A \omega_0\sin(\omega_0t)+2B \omega\cos(\omega_0t)-A \omega_0^2t\cos(\omega_0t)-B \omega_0^2t\sin(\omega_0t))\vec{e_x}
  \]
  Soit
  \begin{align*}
    (-2A \omega_0\sin(\omega_0t)+2B \omega\cos(\omega_0t)-A \omega_0^2t\cos(\omega_0t)-B \omega_0^2t\sin(\omega_0t)) \\
    &+ \omega_0^2(At\cos(\omega_0t)+Bt\sin(\omega_0t))\\
    & =\frac{qE_0}{m}\cos(\omega_0t)
  \end{align*}
  Ce qui implique:
  \[
  \begin{cases}
    2B \omega_0\cos(\omega_0t) =\frac{qE_0}{m}\cos(\omega_0t)\\
    -2A \omega_0\sin(\omega_0t)=0
  \end{cases} \implies
  \begin{cases}
    B=\frac{qE_0}{2m \omega_0}\\
    A=0
  \end{cases}
  \]
  \subQuestion la solution n'est pas bornéee si $E_0\neq 0$
  \Question
  \subQuestion $\ddd{\vec{r}}{t} + \frac{w_0}{Q}\dd{\vec{r}}{t}+ \omega_0^2 \vec{r} =
  \frac{qE_0}{m}\cos(\omega_0t)\vec{e_x}$\\
  Passage en complexe
  \[ i\frac{\omega_0^2}{Q}\vec{r_0}e^{i\phi}= \\frac{q}{m}E_0 \vec{e_x} \]
  \[
    \begin{cases}
      \vec{r_0}= \frac{q}{m \omega_0}E_0 \vec{e_x}\\
      \phi = \frac{-\pi}{2}
    \end{cases}
  \]
  \subQuestion Le mouvement est d'amplitude constante pour rompre la lisaion on
  doit donc avoir:
  \begin{align*}
    r_0 &\ge R\\
    E_0&\ge \frac{m \omega_0^2R}{Qq}
  \end{align*}
\end{Answer}
