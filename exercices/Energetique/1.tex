\begin{Exercise}[title=]
	Une cuve de hauteur $h$ et de section $S$ contient $N\ggg 1$ billes de masse
    $m$ d'énergie identique. Elles rebondissent verticalement et maintienne en
    équilibre un piston de masse $M$.\\
    On enlève le piston à quelle hauteur monte les billes ?
\end{Exercise}
\begin{Answer}
	1ddl => Énergie.\\
	pendant $dt$ le piston reçoit et cède $dp = 2mv dN$.
	On fait l'hypothèse d'une répartition isotrope des particules dans la cuve (
    théorie cinétique des gaz) $n^* = \frac{N}{hS}$ donc $dN = \frac{n^*v dt
      S}{2}=\frac{Nvdt}{2h}$ on a donc:

	\[\frac{dp}{dt} = \frac{Nmv^2}{h} = Mg\]

	Or $E = \frac{1}{2}mv^2 + mgh $ on a donc :
	\[ E = \frac{hg}{2}(M-2m)\]
	On enlève le piston et on prend une bille en $z=0$ qui monte en $h_{max}$ où
    sa vitesse est nulle $h_{max}= \frac{E}{gm} = h\frac{(M-2m)}{2m}$
\end{Answer}
