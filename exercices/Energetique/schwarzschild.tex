\begin{Exercise}[title=Rayon de Schwarzschild]
  \Question Rappelez l'expression de la force d'attraction gravitationelle et
  déduisez en l'expression de l'énergie potentielle de pesanteur.
  \Question Une fusée de masse m est posée à la surface d'une planète de masse M
  et de rayon R . Elle décole avec une vitesse $v_0$
Quelle est la valeur minimale de $v_0$ qui permet à la fusée d'échapper à
l'attraction gravitationelle de la planète ?
  \Question On imagine à présent que la fusée peut décoller à la vitesse de la lumière c .
Exprimez, en fonction de la masse M de la planète, le rayon minimal au-deça duquel la fusée ne
peut plus échapper à l'attraction gravitationelle.\\
\emph{Données} \gdr{G}{6.67e-11}{\m^3\kg^{-1}\s^{-2}} et
\gdr{M_{Terre}}{6e24}{\kg}
\end{Exercise}
\begin{Answer}
 %  \Question ~ $\vec{F_{1\to2}}=\frac{Gm_1m_2}{d^2}\vec{n_{2\to 1}}$ qui dérive de l'énergie
%  potentielle $E=\frac{Gm_1m_2}{d}$
  \Question $v_L = \sqrt{\frac{2GM}{R}}$
  \Question $R_s = \frac{2GM}{c^2}$
\end{Answer}
