\begin{Exercise}[title=Interaction de particules chargée]
  On considère deux particules $A$ (fixe) et $B$ mobile de même masse $m$ de
  charges respective $q_A$ et $q_B$. on considère la force de coulomb entre ces
  deux particules comme étant la seule force en jeu du problème.

  \Question Rappeler son expression.
  \Question Donner l'énergie dont dérive cette force.
  \Question On suppose $q_a=q_b=q$ On lance $B$ vers $A$ avec la vitesse $v_0$
  distante itialement de $A$ À quelle distance minimale $B$ s'approche-t-elle de
  $A$? On pourra s'aider d'un graphe d'énergie potentielle.
  \Question On suppose $q_a=-q_b=q$ Quelle vitesse minimale faut-il donner à $A$
  pour qu'elle puisse s'échapper à l'infini?On pourra s'aider d'un graphe
  d'énergie potentielle.

\end{Exercise}
\begin{Answer}
	\Question $\vec{f}=\frac{q_Aq_B}{4\pi\epsilon_0r^2}dr\vec{u_r}$
	\Question $\delta W =-\d E_p = \vec{f}.\vec{\d r} = \frac{q_Aq_B}{4\pi\epsilon_0r^2}dr
	 \implies E_p = \frac{q_Aq_B}{4\pi\epsilon_0r}+\cancel{C}$
	\Question $E_m = \frac{1}{2}mv_B^2 + \frac{q^2}{4\pi\epsilon_0 r}$
	on a $E_m(v_B=0,d_{min}) = E_m(v_0,a) \implies d_{min}
    =\left(\frac{1}{a}+\frac{2\pi\epsilon_0 m v_0}{q^2}\right)^{-1}$

	\Question Pour des charges opposées : force attractive la vitesse de
    libération est donnée par:
    $ \frac{1}{2}mv_0 -\frac{q^2}{4\pi\epsilon_0 a} = E_m(r_\infty) =0 \implies v_0 =
    \sqrt{\frac{q^2}{2\pi\epsilon_0 a m}}$
\end{Answer}
