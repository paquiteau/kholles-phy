\begin{Exercise}[title=Le marsupilami]
	Le marsupilami est un animal de bande dessinée crée par Franquin aux
    capacité physique remarquable, en particulier grâce à sa queue qui possède
    une force importante. Pour se déplacer, le Marsupilami enroule sa queue
    comme un ressort entre lui et le sol et s'en sert pour se propulser vers le
    haut.\\
    On note \gdr{l_0}{2}{m} :la longueur à vide du ressort équivalent. et
    \gdr{l_m}{50}{cm} la longueur à compression maximale. la masse  de l'animal
    est \gdr{m}{50}{kg}.Il quitte le sol quand la longueur de la queue vaut
    $l_0$.
    \Question Quelle est la constante de raideur de la queue si un saut amène le
    marsupilami à une hauteur \gdr{h}{10}{m} Quelle est sa vitesse lorsque la
    queue quitte le sol?
\end{Exercise}
\begin{Answer}
	BDF : Poids , tension ressort réaction du support. En l'air on ne prend en compte que le poids.
	\Question TEM avec $E_m= E_c+E_p+\underbrace{E_e}_{=0 \text{en l'air}}$
	$k=\frac{2mg(h-l_m)}{(l_m-l_0)^2}=$\SI{4,1}{N\per\m}. et $v_D=\sqrt{2g(h-l_0)}$
\end{Answer}
