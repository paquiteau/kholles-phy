\begin{Exercise}[title=Brisure de symétrie]
  Une masse M$(m)$ est attachée à un ressort de longueur à vide $l_0$ et de
  constante de raideur $k$.
  La masse glisse sans frottement le long de l'axe horizontal et le ressort est
  attaché à une distance R verticale de cet axe.
  \Question En prenant comme référence $E_p(x=0)=0$ montrez que l'énergie
  potentielle à pour expression
  $E_p=\frac{1}{2}\left(\left(\sqrt{R^2+x^2}-l_0\right)^2-(R-l_0)^2\right)$
  \Question Comment déterminer l'existence de positions d'équilibre dans le cas
  d'un système à un degré de liberté ? En séparant les cas $R > l_0$ et $R<l_0$ ,
  étudiez l'existence de positions d'équilibre $x_{eq}$ .
  \Question Comment déterminer la stabilité des positions d'équilibre dans le
  cas d'un système à un degré de liberté ?
  Caractérisez chacune des positions trouvées précédemment.
  \Question Tracez une courbe $x_{eq} = f(R)$ . Pourquoi dit-on d'un tel système
  qu'il présente une bifurcation ?

\end{Exercise}
\begin{Answer}
  \Question la force exercé sur la masse m est $\vec{f}=-k(RM-l_0)\vec{RM}$
  ainsi l'énergie potentielle associée est : (en intégrant entre 0 et x)
  $E_p=\frac{1}{2}\left(\left(\sqrt{R^2+x^2}-l_0\right)^2-(R-l_0)^2\right)$
  \Question on étudie $\deriv{E_p}{x}=\dots=kx \left( 1-\frac{l_0}{\sqrt{R^2+x^2}} \right)$
  les solutions sont:
  $
  \begin{cases}
    x=0 \text{ instable } \\
    x=\pm\sqrt{l_0^2-R^2} \text{ si } R<l_0 \text{ stable}
  \end{cases}
  $
  \Question Un changement du paramètre R conduit à un systeme résolument
  différent (nombre et stabilité des positions d'équilibre)
\end{Answer}
