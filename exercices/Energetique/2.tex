\begin{Exercise}[title=]
	On considère une chaîne de longueur $L$ et de masse linéique $\lambda$ uniforme sur une table. Initialement un quart de sa longueur pend à l'extrémité de la table. Sans frottement pour la retenir, elle  se met à glisser.
	\Question  Déterminer la trajectoire de l'extrémité pendante de la chaîne.
	\Question En considérant les frottements, quelle longueur maximale peut on faire pendre?
\end{Exercise}
\begin{Answer}
	\Question $E_p(z) = -\int_{0}^{z}\lambda z g dz = -\frac{-\lambda g z^2}{2} + E_p(0)$
	Alors en dérivant le TEM :
	\[ \ddot{z} - \omega z = 0 \text{ avec } \omega = \sqrt{\frac{g}{L}} \implies z =\frac{L}{4}\cosh(\omega t) \]
	\Question Modèle de coulomb : $\mu = \frac{\|T\|}{\|N\|} = \frac{x}{L-x} \implies x= \frac{\mu}{1+\mu}\lambda $
\end{Answer}
