\begin{Exercise}[title=Chocs]
  On étudie un choc entre deux particules,$ M_1$ et $M_2$ , de masse $m_1$ et $m_2$, posée sur l'axe horizontal,
  et de vitesses $\vec{v_1}$ et $\vec{v_2}$ suivant l'axe $x$
On suppose le choc élastique:
es particules ne peuvent pas se déformer. Après le choc, elles conservent leur
masse, leur forme et leur vitesse deviennent $\vec{v_1'}$ et $\vec{v_2'}$
\Question Faite une analyse qualitative du problème (symétrie, cas limites
($m_1\gg m_2$))
\Question Écrire les équations de conservation au cours du processus.
\Question Déterminer $\vec{v_1'}$ et $\vec{v_2'}$. Comparer avec l'analyse faites
en 1.
\Question Application une balle de ping-pong est posée sur une balle de tennis à
une hauteur h. on lache les deux balles. A quelle altitude remonte la balle de
ping-pong ?
\end{Exercise}
\begin{Answer}
  \Question pour $m_1\gg m_2$ la balle $M_1$ s'arrete et transmet toutes sa vitesse
  à la balle $M_2$ et s'arrete.
  \Question  Conservation quantité mouvement et énergie cinétique
$  \begin{cases}
    m_1v_1 + m_2v_2 = m_1v_1'+m_2v_2' \\
    m_1v_1^2+m_2v_2^2 = m_1v_1'^2+m_2v_2'^2 \\
  \end{cases}$
  \Question on a donc:
  $\begin{cases}
    v_1' = \frac{m_1-m_2}{m_1+m2}v_1+\frac{2m_2}{m_1+m_2}v_2\\
    v_2' = \frac{2m_1}{m_1+m_2}v_1+\frac{m_2-m_1}{m_1+m_2}v_2
  \end{cases}$
  \Question La balle du dessous touche le sol avant celle du dessus et, dans le
  cas idéal, elle rebondit sans pertes dues aux frottements.
  Il y a alors choc élastique entre deux balles qui ont
  des vitesses opposées.
  $$v_2'=\frac{3-\frac{m_1}{m_2}}{1+\frac{m_1}{m_2}}v $$
\end{Answer}
