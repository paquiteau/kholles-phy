\begin{Exercise}[title=]
	Soit le montage:
	\begin{center}
	\begin{circuitikz}
		\draw (0,0) to[open,v=$e$] (0,3) -- (1,3);
		\draw (1,2.5) -- (1,3.5);
		\draw (1,3.5) to[L,l=$L$] (3,3.5) to[R,l=$r$] (5,3.5);
		\draw (1,2.5) to[C,l_=$C$] (3,2.5) to[C,l_=$C$] (5,2.5);
		\draw (5,3.5) -- (5,2.5);
		\draw (5,3) -- (6,3);
		\draw (6,0) to[open,v=$s$] (6,3);
		\draw (0,0) -- (6,0);
		\draw (3,0) to[vR,l=$R$] (3,2.5);
	\end{circuitikz}
	\end{center}
\Question Déterminer la fonction de transfert à vide de ce filtre.
\Question On souhaite obtenir un réjecteur de fréquence à la fréquence $f_0$. Exprimer cette fréquence en fonction des valeurs des paramètres. En déduire la valeur $R_0$ de la résistance variable $R$ permettant l'obtention d'un tel filtre.
\Question On pose $x=\frac{f}{f_0}$ et $Q= \frac{2\pi f_0L}{r}$.Écrire la fonction de transfert sous la forme $\frac{1}{1+jA}$ en exprimant A en fonction de $x$ et $Q$.
\Question Tracer le diagramme de Bode du filtre.
\Question Déterminer la bande passante à -3dB.
\end{Exercise}
\begin{Answer}
	\Question l'hypothèse de fonctionnement à vide impose l'absence de courant en sortie du filtre, on peux y appliquer le théorème de Millman, on l'applique également entre les deux capacités ($v_A$):
	$v_A = \frac{jRC\omega(e+s)}{1+2jRC\omega}$ et $s=\frac{e+jC\omega(r+jL\omega)v_A}{1+jrC\omega-LC\omega^2}$
	d'ou: $H= \frac{1-rRC^2\omega^2+jRC\omega(2-LC\omega^2)}{1-(L+RrC)C\omega^2+jC\omega(r+2R-RLC\omega^2)}$
	\Question Pour obtenir un réjecteur de fréquence il faut $H=0$ en $\omega = 2\pi f_0$ il faut les partie imaginaire et réelle du numérateur nulle :
	 $\omega_0=\sqrt{\frac{2}{LC}}$ et $R_0= \frac{L}{2rC}$
	\Question on a $A=\frac{2x}{Q(1-x^2)}$
	\Question gain  $\searrow(x=1)\nearrow$  phase décroissante  et discontinue! de 0 à -pi/2 et de pi/2 à 0.
	\Question on veux $\frac{1}{\sqrt{1+A^2}}\geq \frac{1}{\sqrt{2}}$ donc on a une bande :
	$\frac{\pm r+\sqrt{r^2+2L/C}}{L}$
\end{Answer}
