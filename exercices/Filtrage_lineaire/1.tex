\begin{Exercise}[title=]
	On considère le filtre suivant avec : $R_1=R_3=$\SI{1}{k\ohm}, $R_3$=\SI{18}{k\ohm},$C=100nF$
	\Question Prévoir le comportement asymptotique de ce filtre
	\Question Calculer sa fonction de transfert, et la mettre sous forme
    canonique, en explicitant les temps caractéristique et gain statique.
    \Question Établir le diagramme de Bode en précisant les gains en
    décibels pour les pulsations associée au temps caractéristiques.
	\begin{center}
		\begin{circuitikz}
			\draw (0,0) to[open ,v=$\underline{U_e}$] (0,2) to[R,l=$R_1$](2,2) to[C,l_=C](4,2) -- (5,2)to[R,l_=$R_3$,v^<=$\underline{U_s}$](5,0);
			\draw(2,2)--(2,3)to[R,l=$R_2$](4,3)--(4,2);
			\draw(0,0) -- (5,0);
		\end{circuitikz}
	\end{center}
\end{Exercise}
\begin{Answer}
	\Question BF ok , HF ok : Donc Passe tout
	\Question $H(j\omega) = \frac{R_3}{R_1+R_2+R_3} \frac{jR_2C\omega+1}{1+j\frac{R_2(R_1+R_3)}{R_1+R_2+R_3}C\omega}$
	\Question horizontale ; -20db ($\omega_0/10~\to~ \omega_0$) ; horizontale . phase : $0 \to -\pi/2 \to 0 $
\end{Answer}
