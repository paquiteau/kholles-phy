\begin{Exercise}[title=Taille maximale d'une montage]
  Le but est de modéliser la hauteur limite d'une montagne sur une planète de masse $M$ et de rayon $R$. On suppose la montagne de forme cylindrique, de section $S$ et de hauteur $h$ dans le champ gravitationnel uniforme de la planète.


  \Question Rappeler l'expression du champ gravitationnel $g$. Déterminer l'énergie supplémentaire pour rajouter au sommet une masse $\Delta m$, en fonction de $g$ et $h$.
  \Question En déduire la valeur limite de la hauteur $h$ pour laquelle la couche rajoutée au sommet va conduire à faire fondre une couche équivalente à la base de la montagne. L'exprimer en fonction de la chaleur latente de fusion des roches $L_f$. A.N. pour la Terre avec $L_f = 250$ kJ.kg$^{-1}$. Commentaire ?
  \Question En supposant toujours valable le résultat précédent, et en notant $\rho$ la masse volumique uniforme de la planète, en déduire le rayon minimum d'une planète sphérique, défini pour des montagne de hauteur égale au rayon de la planète. Faire l'application numérique avec une masse volumique crustale (de la croûte terrestre) de $3 \times 10^3$ kg.m$^{-3}$.

\end{Exercise}
\begin{Answer}
  Solution
\end{Answer}
