\begin{Exercise}[title=Pendule avec un obstacle]
  On considère un pendule simple modifié. Lorsque $\theta=0$, un clou en $O'$ bloque la partie haute du fil vers la gauche ($OO'=\frac{L}{3}$).

  À la date $t=0$, on abandonne le mobile $M$ sans vitesse initiale, en donnant au fil une inclinaison initiale $\theta(0)=\theta_0>0$. On note $t_1$ la date de la première rencontre du fil avec le clou et $t_2$ la date de la première annulation de la vitesse du mobile pour $\theta <0$.

  Établir l'équation différentielle vérifiée par $\theta$ dans la première phase du mouvement.
    \Question Linéariser cette équation dans l'hypothèse des petites oscillations. Déterminer la durée de la première phase du mouvement sans résoudre l'équation.
    \Question Déterminer la vitesse $v_1^-$ de $M$ à la date $t_1^-$. En déduire la vitesse angulaire $\omega_1^-$ à cette date.
    \Question Le blocage du fil par le clou ne s'accompagne d'aucun transfert énergétique. Déterminer la vitesse $v_1^+$ de $M$ à la date $t_1^+$. En déduire la vitesse angulaire $\omega_1^+$.
    \Question Donner la durée de la deuxième phase.
    \Question Déterminer l'expression de l'angle $\theta$ à la date $t_2$.
    \Question Décrire brièvement la suite du mouvement de ce système et donner l'expression de la période $T$.
    \Question Dresser l'allure du portrait de phase. Conclure.

    \begin{center}
      \begin{tikzpicture}
        \draw[dashed] (0,0) node{$\bullet$}node[left]{$O$} -- ++(0,-+(20:1) node[above]{$\vec{u_\theta}$};
        % partie droite :
        \draw[dashed] (5,0) node{$\bullet$}node[left]{$O$} -- ++(0,-2) node[midway,left]{$\frac{L}{3}$} node{$\bullet$}node[left]{$O'$}-- ++(0,-4) node{$\bullet$};
        \draw (5,0) -- (5,-2) -- ++(-130:4) node[midway,above]{$\frac{2L}{3}$}
        (5,-4) arc(-90:-130:2) node[midway, below]{$\theta_2$};
      \end{tikzpicture}
    \end{center}
  \end{Exercise}
\begin{Answer}
  Solution
\end{Answer}
