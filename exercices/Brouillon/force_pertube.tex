\begin{Exercise}[title=Pertubation d'une force centrale]

  Un satellite ponctuel de masse $m$ est soumis à l'action gravitationnelle d'un astre présentant un défaut de sphéricité. La force résultante est $\vec{f}=-\frac{k}{r^2}(1+\frac{\alpha}{r})\vec{u_r}$ avec $k>0$ et $\frac{|\alpha|}{r} \ll 1$. La trajectoire est voisine d'une ellipse.
    \Question Ré-établir les formules de Binet.
    \Question Montrer que l'équation de la trajectoire est $\frac{1}{r} \approx \frac{p}{1+e \cos(\eta \theta}$ avec $\eta \approx 1 - \frac{\alpha}{2p}$.
    \Question Interpréter : quel est l'angle de rotation du grand-axe de l'ellipse au bout d'une révolution ?
    \Question Traiter de même le cas où $\vec{f}=-\frac{k}{r^2}(1+\frac{\beta}{r^2})\vec{u_r}$ avec $\frac{\beta}{r^2} \ll 1$. On se limite au cas $e \ll 1$.
\end{Exercise}
\begin{Answer}
  DYP 20
\end{Answer}
