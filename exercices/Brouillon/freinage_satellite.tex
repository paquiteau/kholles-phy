\begin{Exercise}[title=Freinage d'un satellite dans l'atmosphère]
  La Terre est assimilée à un astre parfaitement sphérique, et le référentiel géocentrique sera supposé galiléen. Un satellite, dont la trajectoire reste constamment très voisine d'un cercle, subit de la part de la haute atmosphère une force de freinage de norme $Kv^2$.

        Donner une relation simple entre énergie mécanique et énergie cinétique du satellite. Étudier comment évolue la vitesse, expliquer qualitativement pourquoi elle augmente bien que le satellite soit freiné. Étudier comment évolue l'altitude, évaluer approximativement la chute au bout d'une révolution.

\end{Exercise}
\begin{Answer}
DYP 10
\end{Answer}
