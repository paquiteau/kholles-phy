\begin{Exercise}[title=Mouvement d'une comète]
  Une comète a sa trajectoire dans le même plan que celui de la Terre ; sa distance minimale au Soleil (périhélie) est $\frac{R_0}{2} (= 0.5 U.A.)$, la vitesse au périhélie est $2 v_0$ (2 fois celle du centre de la Terre).

  \Question Nature, caractéristiques géométriques de la trajectoire ?
  \Question Cette trajectoire coupe celle de la Terre en 2 points $A$ et $B$ : les situer, déterminer, en module et en direction, la vitesse de la comète en ces 2 points.
  \Question Évaluer, en jours, le temps mis par la comète pour aller de $A$ à $B$.
\end{Exercise}
\begin{Answer}
  DYP 13
\end{Answer}
