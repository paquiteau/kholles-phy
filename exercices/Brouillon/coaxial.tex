\begin{Exercise}[title=Modélisation d'un cable coaxial]
  Un câble coaxial peut être modélisé par un circuit $A_1A_2$ constitué d'une chaîne de $n$ modules identiques comportant chacun trois résistances (de respectivement $\frac{R}{2}$, $2R$ et $\frac{R}{2}$.

  Un dipôle résistor $X_1X_2$ de résistance $2R$ est branché en parallèle à l'extrémité de la chaîne.

  % \begin{center}
  %   \includegraphics[scale=0.75]{coax.jpeg}
  % \end{center}

  \Question Le dipôle est équivalent à un résistor.

  \subQuestion Exprimer, en fonction de $R$, la résistance équivalente $R_1$ dans le cas d'une chaîne ne comportant qu'un seul module.
  \subQuestion Même question pour la résistance $R_2$ dans le cas d'une chaîne à $n=2$ modules.
  \subQuestion En déduire, sur le même principe, la résistance équivalente $R_n$ d'une chaîne à $n$ modules.

  \Question Le dipôle équivalent $A_1A_2$ est alimenté par un générateur de f.é.m. constante $V_0=V_{A_1} - V_{A_2}$.

  \subQuestion Déterminer, en fonction de $V_0$ et $R$, la tension $V_1 = V_{X_1} - V_{X_2}$, aux bornes du résistor $X_1X_2$, dans le cas d'une chaîne ne comportant qu'un seul module.
  \subQuestion Même question pour la tension $V_2$ dans le cas d'une chaîne à $n=2$ modules.
  \subQuestion Idem avec $V_n$ pour $n$ modules.
  \subQuestion En déduire la valeur $V_\infty$ pour une chaîne de longueur infinie.

\end{Exercise}
\begin{Answer}
  Tout en un , chapitre 2
\end{Answer}
