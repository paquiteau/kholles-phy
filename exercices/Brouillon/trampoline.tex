\begin{Exercise}[title=Trampoline]
  On s'intéresse au mouvement d'un enfant sautant sur un trampoline. On modélise le trampoline par une plaque rigide et sans masse posée sur un ressort de raideur $k$. L'origine $O$ de l'axe vertical correspond à la position de la plaque lorsque le trampoline est au repos. On suppose que le système reste toujours vertical.


  \Question Exprimer l'action de la plaque sur l'enfant en fonction de l'action du ressort sur la plaque.

  On suppose qu'à $t=0$, l'enfant est à une hauteur $y(0)=h$ et qu'il se laisse tomber sur le trampoline sans vitesse initiale.

  \Question Exprimer l'énergie potentielle totale de l'enfant (on distinguera les cas $y>0$ et $y<0$). Tracer la courbe $Ep(y)$ en précisant les points remarquables.
  \Question Exprimer l'énergie mécanique $E_m$ de l'enfant et la représenter sur le graphe précédent.
  \Question Déterminer la constante $k$ du ressort pour que la compression maximale du ressort corresponde à $y_{min}$. A.N. : $m=30kg, h=1,0m, y_{min}=-50cm$.

  Dans la suite, on garde cette valeur numérique de $k$.

  \Question Représenter l'allure de la courbe $E_c(y)$.
  \Question Avec quelle vitesse l'enfant arrive-t-il sur le trampoline ? A.N. ?
  \Question Déterminer la valeur de $y$ pour laquelle sa vitesse est maximale. Calculer cette vitesse.
  \Question Après le premier rebond, l'enfant remonte à une hauteur $h'=1,80m$. Montrer que l'enfant a fourni de l'énergie. Exprimer ce travail musculaire. A.N. ? Quelle sera la valeur de la nouvelle position $y_{min}$ correspondant ) la compression maximale du ressort lors du prochain rebond (on suppose qu'il ne fournit pas de travail musculaire entre la remontée à la hauteur $h'$ et le rebond suivant).

\end{Exercise}
\begin{Answer}
  Solution
\end{Answer}
