\begin{Exercise}[title=Chute d'une comète]
  Une météorite possède, à grande distance de la Terre, une trajectoire quasi-rectiligne à la distance $d$ du centre de la Terre, parcourue à la vitesse $\vec{v_0}$.
    \Question Nature de la trajectoire en présence de l'attraction  terrestre ?
    \Question À quelle condition évite-t-elle la Terre ?
    \Question Dans le cas contraire, quel est l'angle de rentrée dans l'atmosphère par rapport à la verticale ?
\end{Exercise}
\begin{Answer}
  \Question mouvement plan car $\vec{r} \wedge \vec{F} = 0 \implies \vec{r} \wedge \vec{v} = \vec{cste} =  r^2 \dot{\theta}\vec{u_z}  = v_0 d$.
  \Question  On se place au point $N$ ou la distance à la terre est minimale. $\vec{v_N} = \dot{r}\vec{u_r}+r_{min}\dot{\theta}\vec{u_\theta} = \frac{v_0b}{r_{min}}\vec{u_\theta}$.
  On utilise alors la conservation de l'énergie : \[E_m = \frac{1}{2}mv_0^2 = \frac{1}{2}mv_{N}^2-\frac{GM_Tm}{r_{min}} = \frac{1}{2}m \frac{v_0^2b^2}{r_{min^2}}-\frac{G M_T m}{r_{min}} \]
  Si $r = R_T$ on a :
 \[
   d = R_T \sqrt{1+\frac{2GM_T}{R_Tv_0^2}} =R_T \sqrt{1+\frac{v_{lib}^2}{v_0^2}}
 \]
 \Question La trajectoire de la comète est une conique d'équation:
 \[
   r = \frac{p}{1+\varepsilon \cos(\theta-\theta_0)}
 \]
 Avec $p = \frac{\sigma_0^2}{mK}$ et $e= \frac{A\sigma_0^2}{mK}$  ou $A$ est une constante d'integration et $\theta_0=0$ ici. on a avec les CL ($r=\infty,\theta=0$) $e=-1$. Donc :
 \[
   \cos{\theta} = \frac{p-r_{lim}}{r_{lim}}
 \]

\end{Answer}
