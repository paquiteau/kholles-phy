\begin{Exercise}[title= Force de gravitation et tunnel terrestre]

  On démontre que pour tout point M de masse m situé à l’intérieur de la Terre, à la distance r du centre O de la terre, l’attraction terrestre est une force agissant en ce point M dirigée vers le centre de la Terre et de valeur $F=-m g_0\frac{r}{R} \overrightarrow{e_r}$ où $R$ est le rayon de la Terre.
  \Question Quelle est l'énergie potentielle de $M$ (en supposant que $E_p=0$ pour $r=0$) ?
  \Question  On considère un tunnel rectiligne AB, d’axe (Hx) ne passant pas par O et traversant la Terre. On note d la distance OH du tunnel au centre de la Terre. Un véhicule assimilé à un point matériel M de masse m glisse sans frottement dans le tunnel. Il Question du point A de la surface terrestre sans vitesse initiale.
  \subQuestion Quelle est sa vitesse maximale $v_m$ au cours du mouvement?  A.N. avec $d=5\times 10^6 m$.
  \subQuestion Exprimer $\bar{HM} = x$ en fonction du temps $t$ par une méthode énergétique. Retrouver l'expression de $v_m$.


\end{Exercise}
\begin{Answer}
  Solution
\end{Answer}
