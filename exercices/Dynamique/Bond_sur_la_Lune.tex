\begin{Exercise}[title=Bond sur la Lune]
	Dans l'album de Tintin "On a marché sur la lune " le Capitaine Haddock est étonné de pouvoir faire un bond beaucoup plus grand que sur terre.
	On assimile le mouvement du Capitaine Haddock à celui de son centre de gravité $M$ de masse $m$ Il saute depuis le sol lunaire avec une vitesse initiale $v_0$ faisant un angle \gdr{\alpha}{30}{\deg}. On note $g_l$ l'accélération de la pesanteur à la surface de la Lune. En l'absence d'atmosphère on peux considérer qu'il n'y a aucune force de frottement.
	\Question Déterminer l'équation du mouvement.
	\Question Déterminer l'expression de la distance horizontale parcourue au cours du saut.
	\Question Sur la Lune la pesanteur est environ 6 fois moins importante que sur Terre. Quelle sera la distance horizontale parcourue par le sauteur sur la Lune si elle est de \gdr{d}{1,5}{m} sur Terre?
\end{Exercise}
\begin{Answer}
	\Question Chute libre classique : $z= \frac{-1}{2}g_l\frac{x^2}{(v_0\cos \alpha)^2}+x\tan\alpha$
	\Question $d=\frac{v_0^2}{g_l}\sin 2\alpha$
	\Question 6 fois plus importante. et meme plus car $v_0$ plus grand.
\end{Answer}
