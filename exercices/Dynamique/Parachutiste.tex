\begin{Exercise}[title=Parachutiste]
	Un commando saute en parachute en territoire ennemi. Son saut se déroule en deux phases. Il commence par sauter de l'avion à l'altitude de croisière $h$ et tombe en chute libre ; puis il ouvre son parachute.
	\Question Proposez un modèle pour décrire le saut.
	On adopte le modèle suivant : tant que son parachute n'est pas ouvert, les frottements peuvent s'exprimer sous la forme $\vec{f_1}= -\alpha_1 \vec{v}$. On suppose que le parachute s'ouvre instantanément et transforme l'expression des frottements en $\vec{f_2} = \alpha_2 v \vec{v}$. Le parachutiste et son paquetage sont assimilés à un point M de masse $m$ .
	\Question Exprimez la vitesse du parachutiste au cours du temps lors de la première phase. Au bout de combien de
	temps a-t-il atteint 95\% de sa vitesse limite ? A quelle altitude le parachutiste se trouve-t-il à ce moment ?
	\Question Dès qu'il a atteint la vitesse $v_lim$ , le parachutiste ouvre son parachute. Exprimez la vitesse du parachutiste en	fonction de la distance parcourue depuis l'ouverture du parachute. Quelle vitesse finale atteint-il?
	\emph{Données numériques:} $\alpha_1$ = \SI{15}{\kg\per\s}, $\alpha_2 = $\SI{57}{\kg\per\m} $m=$\SI{100}{kg}
\end{Exercise}
\begin{Answer}
	On pourra poser $u= \dot{z}^2$ et étudier $\frac{du}{dz}$
\end{Answer}
