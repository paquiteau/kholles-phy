\begin{Exercise}[title=Looping]
	On considère une gouttière $\Gamma$ circulaire, de centre O et de rayon R. On appelle (Oy) l’axe vertical ascendant. La position d’un point $P$ sur $\Gamma$ est repérée par l’angle $\theta$ =$\widehat{\Omega O P}$ ou $\Omega$ est le point le plus bas du cercle.
	Une petite perle P de masse m est enfilée sur la gouttière (liaison bilatérale) qui joue donc le rôle de glissière.
	À l’instant $t = 0$ , on lance $P$ depuis le point $\Omega$ avec une
	vitesse $v_0$ . La perle glisse sans frottements le long de $\Gamma$.
	\Question Exprimer la vitesse de P en un point d’altitude $y$ en
	fonction de $v_0 , g , R $et $y$.
	\Question Étudier alors les différents mouvements possibles de $P$
	suivant les valeurs de $v_0$ .
	\Question
	Déterminer la réaction $\vec{N}$ de la gouttière sur la perle.  Étudier ses variations en fonction de $y$. Commenter.
	\Question On choisit ici $v_0 = 2\sqrt{gR}$ . Déterminer la loi horaire de $\theta$	Quelle est la valeur maximale de $\theta$ ? pour quelle valeur de $t$ est elle atteinte? \\
	\emph{Donnée:} $\displaystyle\int_{0}^{\theta}\frac{d\theta}{\cos(\theta)} = \ln\left| \tan\left( \frac{\theta}{2}+\frac{\pi}{4}\right)\right|$
\end{Exercise}
\begin{Answer}
	\Question TEC : $\frac{1}{2}m v^2 = -mg(y+R)+ \frac{mv_0^2}{2}$
	\Question Tour complet si $v^2 >0$ donc si $v_0 > 2\sqrt{gR}$. sinon oscillation autour de $y_0= $
	\Question PFD pour trouver $\vec{N}=N\vec{e_r}$
\end{Answer}
