\begin{Exercise}[title=Une balle sur un mur...]
	Du haut d'un mur vous lancez trois balles, identiques, avec la même vitesse
    initiale. Vers le haut, le bas (lancés non forcément perpendiculaire au sol)
    et la troisième à l'horizontale.
    \Question Comparez leur temps d'arrivée au sol
	\Question Comparez leurs vitesses d'arrivée au sol.
	\Question (*) Comparez leurs hauteurs de rebondissement (on suppose le chox
    sur le sol élastique)
    \Question (*) Comparez la hauteur maximale (à partir du sol) atteinte par la
    balle lancée vers le haut, à la portée de la balle lancée à l'horizontale
    (distance entre le pied du mur et le point d'impact sur le sol)
\end{Exercise}
\begin{Answer}
  \Question B,Horiz,Haut ( seule la vitesse verticale compte)
  \Question égales
  \Question Horiz : hauteur du mur . les autres hauteur max
  \Question À la limite où la hauteur du mur tend vers 0, la distance atteinte
  par la balle lancée horizontalement  fait de même. À l'inverse,  la porté
  croit comme la racine carré de la hauteur du mur, qui fait l'essentielle de la
  hauteur atteinte par la balle lancée vers le haut. Dans les deux cas, cette
  hauteur est donc supérieur à la portée de la balle lancée à l'horizontale.
\end{Answer}
