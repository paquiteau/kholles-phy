\begin{Exercise}[title=Pourquoi le ciel est-il bleu?]
	Thomson a proposé un modèle d’atome dans lequel chaque électron ($M$) est élastiquement lié à son noyau ($O$) (il est soumis à une force de rappel passant par le centre de l’atome. Nous supposerons que ce électron est freiné par une force de frottement de type fluide proportionnelle à sa vitesse	v et que le centre O de l’atome est fixe dans le référentiel d’étude supposé galiléen. Nous cherchons à étudier l’action d’une onde lumineuse caractérisée par un	champ électrique $\vec{E}(t) = E_0 cos(\omega t)$ , de pulsation $\omega$ (provenant du Soleil) sur un électron	d’un atome de l’atmosphère, représenté à l’aide du modèle de Thomson.
	$k=100N.m$ $ h= 10^{-20}kg/s$
	\Question Écrire l’équation différentielle vectorielle du mouvement de
    l’électron, puis la normaliser.
	\Question Déterminer le régime forcé.
	\Question Simplifier l’expression précédente  en ne considérant que le
    rayonnement visible provenant du Soleil.
	\Question Sachant que l’électron diffuse dans toutes les directions un
    rayonnement dont la puissance	moyenne est proportionnelle au carré de
    l’amplitude de son accélération, expliquer pourquoi le ciel est bleu.
\end{Exercise}
\begin{Answer}
	\Question $\ddot{x}+\frac{\omega_0}{Q}\dot{x}+\omega_0^2x = -\frac{e}{m}E(t)$
	\Question $x(t)=X_m \cos(\omega t+ \phi)$ Avec $X_m =
    \frac{eE_0}{m\omega_0^2}\frac{1}{\sqrt{\left(\frac{\omega^2}{\omega_0^2}-1\right)^2}+\frac{1}{Q^2}\frac{\omega^2}{\omega_0^2}}$
    et $\phi = \frac{\pi}{2}-\arctan Q(\frac{\omega}{\omega_0}-\frac{\omega_0}{\omega})$

	\Question Comparer les valeurs de $\omega_b$ et $\omega_r$ avec $\omega_0$ => $X_m \simeq
    \frac{eE_0}{m\omega_0^2}$ et $ \phi \simeq \pi$
	\Question On a $\ddot{x} \simeq \frac{e\omega^2}{m\omega_0^2}E_0 \cos(\omega t)$ on en déduit que
    le rapport entre bleu et rouge est de 16.$ (2^4)$
\end{Answer}
