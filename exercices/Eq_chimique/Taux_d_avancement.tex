\begin{Exercise}[title=Taux d'avancement]
	\Question Écrire l'équation de la réaction entre le monoxyde d'azote $NO$ et
    le dioxygène conduisant à $N_2O_4$, en prenant comme coefficient
    stoechiométrique les plus petits entiers possibles.
    \Question À l'instant initial, on introduit dans un réacteur fermé : 0,50
    mol de $NO$, 0,70 mol de $O_2$ et 0,20 mol de $N_2O_4$ . quel est le réactif
    limitant ?
    \Question Au bout d'un temps $t$ il reste 0,30 mol de $NO$.
		\subQuestion Quel est alors l'avancement chimique de la réaction?
		\subQuestion Quel est le taux d'avancement ?
		\subQuestion Déterminer la quantité de matière de chacun des constituants du système.
	\Question A quel valeur d'avancement correspond le taux d'avancement de 90\%
	\Question Pour 0,50 mol de $NO$ introduite, combien faut-il introduire de $O_2$ pour être dans les proportions stœchiométriques?
    \end{Exercise}
\begin{Answer}
\emph{TD PCSI chimie T1 }
	\Question $2NO + O_2 = N_2O_4$
	\Question
	\begin{center}
			\begin{tabular}{l|l|l|l}

		 	    &\multicolumn{3}{c}{$2NO + O_2 = N_2O_4$}\\
			\hline
	 		t=0 & 0,50       & 0,70       & 0,20 \\
	  		t   & 0,50-$\xi$ & 0,70-$\xi$ & $\xi$ \\
			\hline
		\end{tabular}
	\end{center}
	Si NO est limitant : $ 0,5 - 2 \xi_{max} = 0 $ et $\xi_{max} = 0,25$ mol \\
	Si O$_2$ est limitant : $ 0,7 - 2 \xi_{max} = 0 $ et $\xi_{max} = 0,7 > 0,25 $ mol \\
	Donc NO est le réactif limitant.
	\Question Au bout d'un temps t :  $0,50 - 2 \xi = 0,30$ donc $\xi = 0,10$ mol .
    et le taux d'avancement $\tau = \xi /\xi_{max}=0,10/0,25=40 \% $
    On a l'EF : n(O$_2$) = 0,60 mol et n(N$_2$O$_4$) = 0,30 mol .
	\Question $\xi = \tau * \xi_{max}  = 0,23$ mol
	\Question Il faut introduire n(O$_2$)$_0$= n(NO)$_0$/2 = 0,25 mol .
\end{Answer}
