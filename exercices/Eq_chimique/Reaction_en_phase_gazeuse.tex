\begin{Exercise}[title=Réaction en phase gazeuse]
On considère la réaction suivante en phase gazeuse  dans un réacteur fermé à T et V constants :
\[2 N_2O_5 = 4 NO_2 + O_2 \]
On introduit le réactif seul. Les gaz sont assimilés à des gaz parfaits.

	\Question Exprimer la pression totale P à un instant t en fonction de la pression initiale $P_0$ et du taux de décomposion $\alpha$ de $N_2O_5$ définie comme le nombre de mole de $N_2O_5$ ayant réagi sur le nombre de moles initial.
	\Question À l'équilibre, la pression totale est égale à $ 2 P_0$. La  réaction est elle totale ? Déterminer les fractions molaires des différentes composés à l'équilibre.
	\Question Sachant que $P_0 = $ 2 bar, calculer la valeur de la constante d'équilibre K à la température de travail.
\end{Exercise}
\begin{Answer}
	\Question
		\begin{tabular}{l||c|c|c||c|c}
		& \multicolumn{3}{c}{2N$_2$O$_5$ = 4NO$_2$ + O$_2$} & n$_{gaz}$ & pression \\
		\hline
		 t=0 		&  n$_0$ 			&   0      		& 0    			& n$_0$     &  P$_0$ \\
		 t			&  n$_0-2\xi$		& $4\xi$   		&$\xi$			& n$_0+3\xi$&  P$_0$ + n$_0RT/V$ \\
		 			& n$_0(1-\alpha$ 	& 2n$_0\alpha$ 	&n$_0\alpha/2$ 	& n$_0(1+3\alpha/2)$& $(1+3\alpha/2)P_0$\\
		\end{tabular}
		avec $\alpha= 2\xi / n_0$
	\Question
	À l'équilibre $2P_0 = (1+3\alpha/2)P_0$ donc $\alpha = 2/3 < 1 $. La réaction n'est pas totale.
	$n=2n_0$ donc : $x(N_2O_4)=1/6$ ; $x(O_2)=1/6$ $x(NO2)=2/3$
	\Question
	\[ K = \frac{P(O_2)P(NO_2)^4}{P(N_2O_5)^2P_0^3}= \dots = 76\]
\end{Answer}
