\begin{Exercise}[title=Évolution et équilibre]
\ExePart[title={Réaction acide-base}]
 On considère un système évoluant selon la réaction d'équation bilan: $CH_3COOH_{(aq)} +F^-_{(aq)} = CH_3COO^- + HF_{(aq)}$ de constante d'équilibre $K=10^{-1,6}$ à $298K$
Pour chacun des mélanges initiaux suivant déterminer le sens de l'évolution spontanée et l'état d'équilibre. Indiquer à chaque fois si la réaction est totale ou équilibrée:
	\Question $[CH_3COOH]_0= [F^-]_0 = 0,10 ~ mol/L$ et $[HF]_0=[CH_3COO^-]_0 = 0 ~ mol/L$
	\Question $ [CH_3COOH]_0= [F^-]_0 =[HF]_0=[CH_3COO^-]_0 =  0,10 ~ mol/L $
\ExePart[title={Réaction d'oxydo-réduction}]
	Les ions cuivres (II) $Cu^{2+}_{(aq)}$ peuvent réagir en solution aqueuse avec du cuivre solide $Cu_(s)$ pour donner des ions cuivre (I) $Cu^{2+}_{(aq)}$.
	\Question Écrire l'équation de réaction en utilisant les plus petits coefficients stœchiométriques entiers possibles.
	La constante d'équilibre de cette réaction vaut $K=91$ à  la température de travail.
	\Question Déterminer le sens d'évolution d'un systeme obtenue en mélangeant du cuivre solide en excès avec 40,0mL d'une solution de nitrate de cuivre(I) à $C_1= $1,0. 10$^{-3}$ mol/L et 10,0 mL d'une solution de sulfate de cuivre (II) à $C_2= $2,0.10$^{-3}$ mol/L
	\Question Déterminer la composition du système à l'état final. La réaction est-elle totale?
\end{Exercise}
\begin{Answer}
\ExePart[title={Réaction Acide-base}]
~\\
\Question $Q_0$= 0 <K . la réaction est dans le sens direct. \\
\begin{center}
		\begin{tabular}{l||p{2,2cm}|c|p{2,2cm}|c}
	   & \multicolumn{4}{c}{$CH_3COOH +F^- = CH3COO^- + HF $}\\
	   \hline
	EI & 0,10 	& 	0,10	& 0	& 0 \\
	EF & 0,10-x & 	0,10-x 	& x	& x \\
	\end{tabular}
\end{center}
D'apres la LAM on a $Q_{eq}=K = \frac{x^2}{(0,1-x)^2}=10^{-1,6}$ on résoud et on a :
$x= 1,4. 10^{-2}$ mol/L d'où les concentrations à l'équilibre :

$[CH_3COOH] =[F^-] = 8,6.10^-2$ mol/L et $[CH_3COO^-] =[HF] = 1,4.10^-2$
\Question
$Q_0$= 1 >K . la réaction est dans le sens indirect. \\
\begin{center}
		\begin{tabular}{l||p{2,2cm}|c|p{2,2cm}|c}
	   & \multicolumn{4}{l}{$CH_3COOH +F^- ~~ = CH3COO^- + HF $} \\
	   \hline
	EI & 0,10 	& 	0,10	& 0,10	& 0,10  \\
	EF & 0,10-x & 	0,10-x 	& 0,10+x& 0,10+x \\
	\end{tabular}
\end{center}

D'apres la LAM on a $Q_{eq}=K = \frac{(0,1+x)^2}{(0,1-x)^2}=10^{-1,6}$ on résoud et on a :
$x= -7,3. 10^{-2}$ mol/L d'où les concentrations à l'équilibre : ( $x \neq x_max$) $[CH_3COOH] =[F^-] = 1,7.10^-1$ mol/L et $[CH_3COO^-] =[HF] = 2,7.10^-2$
\emph{K << 1 il a tjr plus de réactif que de produits}
\ExePart[title={Réaction oxydo-réduction}]
\Question $Cu^{2+}+Cu = 2 Cu^+$
\Question calculons les concentrations initiales dans le mélange: 
$[Cu^{2+}]_0=4,0.10^{-4}$mol/L; 
$[Cu^{+}]_0=8,0.10^{-4}$mol/L;
$Q=1,6.10^{-3} < K $ sens direct.
\Question
	\begin{center}
		\begin{tabular}{l||p{2,2cm}|c|p{2,3cm}}
	   & \multicolumn{3}{c}{$Cu^{2+} ~~ + ~ Cu = 2Cu^+$} \\
	   \hline
	EI & $4,0.10^{-4}$ 	& 	excès	& $8,0.10^{-4}$  \\
	EF & $4,0.10^{-4}-x$ & 	excès 	& $8,0.10^{-4}+2x$\\
	\end{tabular}
	\end{center}
	On a d'après la LAM :
	\[ Q_e =K= \frac{(8,0.10^{-4}+2x)^2}{4,0.10^{-4}-x} = 91 \]
	 => $x_{eq} = 4,0.10^{-4}$ . alors $[Cu^{+}]_{eq}=1,6.10^{-3}$mol/L et $[Cu^{2+}]= \epsilon = 2,8.10^{-8} $
	 \emph{espèce dissoute elle ne disparait pas totalement, $\epsilon$ se trouve avec la LAM}
	 Réaction totale.
\end{Answer}
