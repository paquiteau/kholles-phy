\begin{Exercise}[title=piston en rotation autour d'un axe]
	Un cylindre calorifugé est mis en rotation de manière pro-
	gressive à partir de la vitesse nulle jusqu’à la vitesse angulaire $\omega$ (qui restera constante) autour d’un axe vertical.
	Un piston mobile de masse m et de section S glisse sans
	frottement à l’intérieur du cylindre ; il emprisonne une
	quantité d’air initialement caractérisée par les conditions
	${P_0 , T_0 , V_0 }$. L’air sera considéré comme un gaz parfait.
	\Question Déterminer la pression finale $P_f$ du gaz si l’on admet qu’il a subi une transformation quasi-statique réversible lorsque le piston s’est déplacé de sa position initiale caractérisée par $r_0$ jusqu’à sa position d’équilibre caractérisée par $r_f$ .
	\Question En déduire la vitesse angulaire $\omega$ et la température finale $T_f$ du gaz.
	\emph{Données : \gdr{P_0}{1013}{hPa} \gdr{S}{10}{cm^2}; \gdr{r_0}{10}{cm}; \gdr{r_f}{12}{cm};\gdr{m}{1}{kg}; \gdr{T_0}{293}{K} et\gdr{\gamma}{1.4}{}}
	\Question Application numérique de $P_f, T_f $ et $\omega$
\end{Exercise}
\begin{Answer}
	\Question $P_f= P_0 \left(\frac{r_0}{r_f}\right)^\gamma$
	\Question $T_f = T_0 \left(\frac{r_0}{r_f}\right)^{\gamma-1}$
	\Question \gdr{P_f}{0,785}{bar} ; \gdr{T_f}{272}{K}; \gdr{\omega}{13,8}{rad\per\s}. (obtenue avec premier principe complet).
\end{Answer}
