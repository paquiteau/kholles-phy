\begin{Exercise}[title=Méthode de Rückhardt]
	Soit un gaz parfait occupant un volume $V_0$ (de l’ordre de 10 L) d’un récipient surmonté d’un tube de verre (d’une longueur de 50 cm environ) et de section faible S.
	Considérons une bille d’acier, de masse $m$, susceptible de glisser le long du tube de verre.
	Cette bille se trouve en équilibre mécanique en un point O.
	\Question Quelle est la pression P 0 du gaz à l’intérieur du récipient (en notant $P_0$ la pression atmosphérique et $g$ l’intensité du champ de pesanteur) ?
	\Question On écarte la bille de $x_0$ , à partir de O, à l’instant t = 0, et on la lâche sans lui communiquer de vitesse initiale. $x_0$ est suffisamment faible pour avoir $x_0S \ll V_0$.
	\subQuestion En négligeant les frottements (‘fluides’ comme ‘solide’), établir l’équation du mouvement
	vérifiée par $x(t)$. On supposera que la transformation du gaz est adiabatique quasi-statique. Indiquer la période T des oscillations et en déduire l’expression de $\gamma$ en fonction des paramètres.
	\subQuestion Indiquer la période $T$ des oscillations et en déduire l’expression de $\gamma$ en fonction des paramètres expérimentaux
	\subQuestion Exprimer $x$ en fonction de $t$.
\end{Exercise}
\begin{Answer}
	\Question $P_0'  = P_0 + \frac{mg}{S}$
	\Question $PV^\gamma = C^{ste}$ car adiabatique (méca avant thermo) et réversible. On prend la différentielle logarithmique :
	\[\d P \simeq \frac{-\gamma P_0' S}{V_0}x(t)\]
	On applique le PFD à la bille , et on se ramène à $\d P$ ce qui conduit à : $\ddot{x} +\omega_0^2 x = 0$ avec $\omega_0^2 = \frac{\gamma P_0' S^2}{mV_0}$
	Donc \[\gamma = \frac{4\pi^2 mV_0}{T^2P_0'S^2}\]
\end{Answer}
