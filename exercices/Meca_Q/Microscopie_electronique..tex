\begin{Exercise}[title=Microscopie électronique.]
  \Question Un microscope optique ne peut révéler des détails plus petits que l’ordre de grandeur de la longueur d’onde de la lumière visible. Expliquer pourquoi et  donner une valeur numérique typique.
\Question La longueur d’onde de de Broglie pour des électrons accélérés sous une tension de 100 V, donc ayant acquis une énergie cinétique de 100 eV, est beaucoup plus courte. Quelle est sa valeur ?\\
$m_e=$\SI{9.11e-31}{kg} $e$=\SI{1.6e-19}{C} $h=$\SI{6.62e-34}{J.s}
\Question Dans certains appareils, l’énergie cinétique atteint
t 100 keV et la longueur d’onde obtenue est alors de l’ordre de 1 pm = $10^-12$ m. Pour évaluer cette longueur d’onde, montrer que l’on doit avoir recours aux formules de mécanique relativiste.
L’énergie cinétique répond alors à l’expression (non exigible) : $E_c= (\gamma-1)mc^2$ avec $c=$\SI{3e8}{\m\per\s} et $\gamma=\frac{1}{\sqrt{1-\frac{v^2}{c^2}}}$ mettant en jeu le rapport $v/c$ de la vitesse de la particule à celle de la lumière. Calculer la longueur d’onde  des électrons, sachant que la quantité de mouvement s’écrit $p =\gamma mv$  pour le cas relativiste
\end{Exercise}
\begin{Answer}
\Question diffraction $\lambda$ 400 à 800 nm
\Question $\lambda =0.1$ nm
\Question par la méca classique v=0.6c > 0.1c non valide : $\gamma$ =1.2 ,v =\SI{1.66e8}{\m\per\s}; $\lambda=3.7pm$
\end{Answer}
