\begin{Exercise}[title=Longueur d'onde de Broglie]
  \Question Calculer la longueur d'onde associée au mouvement:
  \subQuestion de la Terre autour du Soleil (masse \SI{6e24}{kg} , vitesse \SI{3e4}{\m\per\s}
  \subQuestion d'un homme marchant dans la rue (\SI{70}{kg} ,\SI{5}{\km\per\hour})
  \subQuestion d'un grain de poussière (\SI{1e-15}{kg}, \SI{1}{\mm\per\s})
  \Question Diffraction par des neutrons. On envoie un faisceau de neutrons
  (masse $m=$ \SI{1.67e-27}{kg}) , d'énergie $E$ sur une chaîne de noyau
  monoatomique. Le jet est perpendiculaire à la chaîne et on note $a$ la
  distance entre deux atomes. ce jet de neutrons va diffracter comme un faisceau
  lumineux selon des angles donnés par la relation : $\sin\theta_n = n\lambda_{dB}/a$.
  \subQuestion Déterminer l'expression de la longueur d'onde de de Broglie
  associée au mouvement des neutrons.
  \subQuestion On détecte les neutrons dans une direction $\theta$ par aux neutrons
  incidents. Quelle est l'allure du flux de neutrons détecté lorsqu'on fait varier E ?
  \subQuestion Le flux présente un maximum pour $E=E_1$ et n'a pas d'autre
  résonances à énergie inférieur à $E_1$ En déduire $a$ avec $\theta$ =
  \SI{30}{\celsius} et $E_1=$\SI{1.3e-20}{J}
  \Question Optique atomique:
  \subQuestion La lithographie est la technique actuellement utilisé pour
  fabriquer les composants semi conducteurs tels que les puces électroniques. Le
  détail le plus petit est alors de l'ordre de grandeurs de la longueur d'onde
  utilisé pour la gravure.
  Plusieurs équipe  de recherche dans le monde cherchent à remplacer la lumière
  par des atomes, notamment l'atome d'hélium ($m$=\SI{6.7e-27}{kg}).\\
  Calculer la vitesse moyenne d'un gaz d'hélium à température ambiante. En
  déduire la longueur d'onde de de Broglie? Conclure.\\
  \emph{On donne l'énergie thermique d'une particule monoatomique
    $E=\frac{3}{2}k_B.T$ avec\\ $k_B$ =\SI{1.38e-23}{\J\per\K} la constante de
    Bolztmann}
  \subQuestion Les techniques de refroidissement d'atome par laser ( prix Nobel
  de physique 1997) permettent aujourd'hui d'atteindre des températures de
  l'ordre de \SI{100}{nK}. Calculer la longueur de de Broglie ainsi obtenue.
\end{Exercise}
\begin{Answer}
  \Question $\lambda=\frac{h}{p}$
  \subQuestion \SI{3.3e-63}{m}
  \subQuestion \SI{6e-36}{m}
  \subQuestion \SI{6e-16}{m}
  \Question
  \subQuestion $v= \sqrt{\frac{2E}{m}} $ donc $\lambda=\frac{h}{\sqrt{2Em}}$
  \subQuestion courbe périodique flux= f(E) , maximum locaux sur les $E_i$.
  \subQuestion $a= \frac{2h}{\sqrt{2Em}} \simeq 10^{-10}$m
  \Question
  \subQuestion $v=\sqrt{\frac{3k_BT}{m}}$
  \subQuestion $\lambda= \frac{h}{\sqrt{3mk_BT}} \simeq 10^{-10}$m
\end{Answer}
