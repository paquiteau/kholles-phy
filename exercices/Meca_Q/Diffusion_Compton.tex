\begin{Exercise}[title=Diffusion Compton]
  L’américain Arthur Compton a réalisé en 1923 l’expérience suivante. Il a envoyé des rayons X durs (c’est-à-dire une onde électromagnétique de fréquence élevée, donc de très faible longueur d’onde $\lambda$ typiquement de 1 pm à 1 nm) sur une mince feuille de graphite. Il a observé que l’onde était diffusée (déviée) dans la direction $\theta$ vérifiant la relation : $\lambda' - \lambda = \frac{h}{mc}(1-cos(\theta))$ où $\lambda'$
  est  la  longueur  d’onde  diffusée, $m =$\SI{9,1e-31}{kg}  la  masse  de l’électron, $h =$ \SI{6,6e-34}{J.s} la constante de Planck et $c =$ \SI{3.108}{\m.\s^{-1}} la vitesse de la lumière.

  Cette relation peut être établie théoriquement par un bilan d’énergie et de quantité de mouvement écrit dans le cadre de la mécanique relativiste. L’expérience peut être interprétée en termes corpusculaires, mais pas de manière ondulatoire, vu le changement de fréquence du rayonnement. Tout comme l’effet photoélectrique, elle nécessite donc la notion de photon. 

  \Question Montrer que la quantité $\frac{h}{mc}$ est homogène à une longueur et la calculer.
  \Question  Pourquoi cette expérience est-elle spécialement intéressante pour des rayons X ?
  \Question  Comment évolue l’énergie du photon dans cette expérience ? Que se passe-t-il ?
  \Question  Pour des rayons X incidents avec $\lambda$ = \SI{7,08e-11}{m}, Compton a observé des rayons X diffusés à $\theta = 90^o$. Quelle est leur longueur d’onde ?
  \Question  Quelle est l’énergie $\Delta E$ perdue par le photon ? Qu’en déduire sachant qu’une énergie d’ionisation est de l’ordre de la dizaine d'eV ?
\end{Exercise}
\begin{Answer}
  \Question $mc =[MLT^{-1}]$ et $h=[ML^2T^{-1}]$
  \Question $\frac{h}{mc}$ correspond au maximum de variation observable =\SI{0.024e-10}{m} variation de l'ordre de grandeur dans les rayons X.
  \Question L'énergie du photon diminue. 
  \Question $\lambda'=$\SI{7,3e-11}{m}
  \Question $\Delta E=600 eV$
\end{Answer}
