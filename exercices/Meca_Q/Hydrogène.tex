\begin{Exercise}[title=Dimension de l'atome d'hydrogène]
  On considère un atome d’hydrogène sphérique, de taille caractéristique a. Cet espace est supposé imposer un confinement à l’électron de cet atome. On admet l’approximation suivante pour l’énergie de l’électron dans l’atome :
  \[
    E = \frac{\hbar^2}{2ma^2}+\frac{-e^2}{4\pi \epsilon_0 a}
  \]
  Avec $ \epsilon_0 =$\SI{8,85e-12}{F/m} , $\hbar = \frac{h}{2 \pi}$ et $h=$\SI{6,62e-34}{J.s} et $m=$\SI{9,11e-31}{kg}.

  \Question En employant la relation d’Heisenberg, justifier que l’énergie cinétique minimale de l’électron correspond au premier terme de l’expression. À quoi correspond le deuxième terme ?
  \Question Déterminer la valeur $a_{min}$ du paramètre a qui minimise l’expression de l’énergie $E$. Calculer sa valeur numérique, qui donne l’ordre de grandeur de la taille de l’atome d’hydrogène.
  \Question Quelle est la valeur minimale de l’expression approchée de $E$ ? Dans l’état fondamental, on mesure expérimentalement $E_o = -13,6 eV$. Comparer.
  \Question
  En mécanique classique, pour un électron en orbite circulaire de rayon r autour du noyau, on trouve une énergie mécanique :
  \[
    E = \frac{-e^2}{8\pi\epsilon_0 r}
  \]
  De plus, l’électromagnétisme permet de montrer que l’électron en mouvement produit lors un champ électromagnétique dont le rayonnement se traduit par une très rapide perte d’énergie du système. Interpréter  la  phrase  suivante : `` C’est  l’inégalité  d’Heisenberg  qui est  à  la  base  de  la  stabilité  des atomes''
\end{Exercise}
\begin{Answer}
  \Question $\Delta h \Delta p > \hbar \implies p > \frac{h}{a}$ On a donc $E_c= p^2/(2m) =\hbar^2/(2ma^2)$
\Question Résoudre $\frac{\d E }{\d a} = 0$ on a $a_{min}=$\SI{5,3e-11}{m}.
\Question $E_{min} = $\SI{-2,1e-18}{J} = -14 eV
\Question
 l’électron ne peut pas se trouver sur une orbite de rayon trop petit, sous peine d’acquérir une impulsion importante, qui fait croître son énergie cinétique. L’énergie de l’état fondamental est obtenue en recherchant le meilleur compromis possible entre énergie cinétique et énergie potentielle, de façon à obtenir le minimum de l’énergie totale
\end{Answer}
