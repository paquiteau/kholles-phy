\begin{Exercise}[title=Oscillateur Parfait]
	Déterminer i(t):
	\Question A partir des lois de comportements des dipôles
	\Question Avec un raisonnement énergétique.
	\emph{A l'instant initial le condensateur est complétement chargé avec la charge $Q$}
	\begin{center}
		\begin{circuitikz}
			\draw(0,0) to[L] (0,2) -- (1,2) to[C,i=$i$](1,0) -- (0,0);
		\end{circuitikz}
	\end{center}
\end{Exercise}
\begin{Answer}
\Question	on a $u= -L\dd{i}{t}$ et $i =  C\dd{u}{t} $ donc $i = -LC \ddd{i}{t}$
\Question $E = E_C + E_L = \frac{1}{2}Li^2 + \frac{1}{2}\frac{q^2}{C}$ on dérive et tout va bien.
\end{Answer}
