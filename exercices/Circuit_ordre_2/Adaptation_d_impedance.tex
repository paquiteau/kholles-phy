\begin{Exercise}[title=Adaptation d'impédance]
	On utilise un générateur de tension réel, de force électromotrice $E(t) = E_0 \sin\omega t$  et de résistance interne $R_0$.On branche à ses bornes un dipôle d'impédance	$Z = R + iL\omega$; on notera $U(t)$ la tension à ses bornes.Dans l'ensemble de l'exercice, on se placera dans le cadre de l'ARQS.
	\Question Exprimer la forme générale de l'intensité	I(t) qui parcourt le circuit, puis exprimez l'amplitude, la valeur effcace et le déphasage de l'intensité.
	\Question Exprimez la puissance $\mathcal{P}$ moyenne reçue par l'impédance	Z .
	\Question On suppose dans cette question que le dipôle est une résistance pure. Montrez que, pour une certaine valeur $R_m$ de la résistance, la puissance reçue par l'impédance présente un maximum.
	Application numérique $E_{eff}=$ 220 V , $R_0$ = \SI{10}{\ohm},$\frac{\omega}{2\pi}	=$ 50 Hz
	\Question On suppose à présent que l'inductance a une valeur fixée
	L = 0.1H .	Montrez que la puissance dissipée par l'impédance totale est toujours inférieure à celle dissipée en l'absence d'inductance. Déterminez la nouvelle résistance $R_m'$ qui maximise la puissance dissipée.
\end{Exercise}
\begin{Answer}
	\Question $I(t)=I_0\sin(\omega t+\varphi)$  (loi des mailles)
	$\underline{E}(t)=\underline{I}(t)Z +\underline{I}(t)R \Leftrightarrow E_0 \ell{i(\omega t +\pi/2)} = I_0 \e^{i(\omega t +\pi/2+\varphi_I)}(R_0+R+iL\omega)$\\
	donc $I_0\ell{i\phi_I}=\frac{E_0}{R+R_0+iL\omega}$\\
	d'où : $I_0 = \frac{E_0}{\sqrt{(R+R_0)^2+L^2\omega^2}}$
	\Question Par définition :
	$\mathcal{P} = <I(t)U(t)>= \frac{I_0U_0}{2}cos(\varphi_U -\varphi_I)$\\
	d'où $U_0=I_0\sqrt{R^2+L^2 \omega^2}$ \\
	on trouve : $\cos(\varphi_U-\varphi_I) = \frac{R}{\sqrt{R^2+L^2\omega^2}} et \sin(\varphi_U-\varphi_I)= \frac{L\omega}{\sqrt{R^2+L^2\omega^2}}$.
	 donc : $\mathcal{P}=RI_0^2 /2 = \frac{RE_{eff}^2}{(R+R_0)^2+L^2\omega^2}$
	\Question Pour L=0, il faut $R=R_0$ et alors $\mathcal{P}=1,2$ MW.
	\Question il faut $R= \sqrt{L^2\omega^2+R_0^2} $ = \SI{33}{\ohm}
\end{Answer}
