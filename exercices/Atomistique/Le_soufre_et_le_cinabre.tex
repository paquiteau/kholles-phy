\begin{Exercise}[title=Le soufre et le cinabre]
	\ExePart[title=Le soufre]
	Le soufre est connu depuis l’antiquité, car on peut le trouver à l’état natif au voisinage des zones volcaniques. C’est vers la fin des années 1770 qu’Antoine Lavoisier attribue au soufre le statut d’élément chimique.
	Le corps simple se présente sous de nombreuses formes selon son mode d’obtention: cristaux ou
	aiguilles jaune pâle, poudre jaune mat (fleur de soufre)...\\
	Le numéro atomique du soufre est $Z=16$
	\Question Déterminer la position du soufre dans le tableau périodique (ligne, colonne).
	\Question Combien un atome de soufre admet-	il d’électrons célibataires? d’électrons de valence?
	\Question Quel est le numéro atomique de l’élément situé juste au-dessus du soufre dans la classification ? Quel est cet élément	? Comparer son électronégativité
	à celle du soufre.
	\Question Parmi les	éléments soufre, chlore et argon, l’un d’eux n’a pas de valeur d’électronégativité de 	Pauling connue,	lequel	?Pour les autres on relève les valeurs 2,58 et 3,16. Attribuer à chaque élément son électronégativité.
	\ExePart[title=le cinabre]
	Le cinabre est un minéral d’origine volcanique de formule HgS, se présentant sous la forme de cristaux rouge vif. Il s’agit du minerai de mercure le plus important.
	On rappelle que le mercure (Hg) fait partie du bloc $d$ de la classification périodique des éléments.
	\Question Si on admet la liaison chimique comme ionique, quels sont les ions constituant le cinabre HgS Pour répondre à cette question, on indique que l’ion du soufre possède une configuration électronique identique à celle du gaz noble de plus proche numéro atomique, mais ce n’est pas le cas pour le mercure.
	\Question Combien le bloc $d$ comporte-t-il de colonne? Justifier ce nombre de colonnes en introduisant les nombres quantiques appropriés.
	\Question Sachant que l’ion du mercure identifié à la question précédente ne comporte aucun électron célibataire dans sa configuration électronique, en déduire dans quelle colonne du tableau périodique se situe le mercure.
	\Question Sachant que le mercure est situé dans la 6\up{e} période de la classification, déterminer le numéro atomique du mercure.
\end{Exercise}
\begin{Answer}
	\ExePart[title=Le soufre]
	\Question $S(Z=16)$ : $1s^2 2s^2 2p^6 3s^2 3p^4$ ,$n_{max} = 3$donc S est dans la 3ème période de la classification.
	période 3, colonne 16 (2+10+4)
	\Question Les orbitales pleines ne contiennent que des électrons appariés. Les électrons célibataires se
	trouvent donc dans les orbitales incomplètes, à savoir ici
	3p. On applique la règle de Hund qui stipule que les électrons tendent à se placer à spins parallèles dans des OA dégénérées, ce qui donne la répartition suivante : $\uparrow\downarrow ~~\uparrow~~\uparrow$ donc on a 2 électrons célibataire. Le soufre possède six électrons de valence. et dix électrons de cœur.
	\Question c'est l'Oxygène ( Z=8) plus électronégatif (augmente de bas en haut et de gauche à droite)
	\Question , Ar est un gaz noble , pas d'électronégativé définie. $\chi (S)=2.58$ et $\chi(Cl)=3.16$.
	\ExePart[title=le cinabre]
	\Question Le soufre a des propriétés similaires à l’oxygène. Il a une électronégativité assez élevée et tend à adopter la configuration électronique du gaz noble qui le suit (l’argon), en capturant deux électrons, soit l'ion $S^{2-}$ et donc on a pour le mercure $Hg^{2+}$
	\Question 10 colonnes dans le bloc d ( pour $l=2$ $-l\leq m_l\leq l$)
	\Question colonne 12 (10\up{e} colonne du bloc $d$)
	\Question  $Hg:1s^2 2s^2 2p^6 3s^2 3p^6 4s^2 3d^{10} 4p^6 5s^2 4d^{10} 5p^6 6s^2 4f^{14} 5d^{10}$  donc Z=80

\end{Answer}
