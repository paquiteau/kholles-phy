\begin{Exercise}[title=Quelques questions autour du tableau périodique]
	\Question Sachant que le polonium (Po) appartient à la colonne 16 et à la sixième période, quel est son
	numéro atomique	?
	\Question Le palladium (Pd) est situé sous le nickel (Ni, Z=28) dans le tableau périodique. En déduire son
	numéro atomique.
	\Question 	Quel est l’ion le plus courant issu du rubidium
	(Rb,Z=37)?
	\Question Quel est le numéro atomique de l’élément alcalino-terreux succédant au baryum (Ba, Z=56)?
	\Question Quel serait le numéro atomique du premier élément d’un éventuel bloc g ? Combien de colonnes
	comporterait ce bloc ? Où faudrait-il le situer dans le tableau périodique? Pourquoi ne figure-t-il sur aucune classification périodique ?
\end{Exercise}
\begin{Answer}
	\Question Les renseignements fournis conduisent à trouver que la configuration électronique se termine
	par $p^4$(la colonne 16 est la 4ème du bloc $p$) et que le nombre quantique principal le plus élevé que la
	configuration possède est $n_{max}=6$(période 6)
	: la configuration contient donc $6s^2$ et ne contient pas $7s^2$ En utilisant la règle de Klechkowski, on rempli les OA jusqu'au premier $p^4$ qui suit $6s^2$ ce qui donne:
	\[1s^2 2s^2 2p^6 3s^2 3p^6 4s^2 3d^{10} 4p^6 5s^2 4d^{10} 5p^6 6s^2 4f^{14} 5d^{10}6p^4\]soit $Z=84$
	\Question $Ni(Z=28)$ : $1s^2 2s^2 2p^6 3s^2 3p^6 4s^2 3d^8$. le palladium est situé sous le nickel, donc sa configuration finit par $4d^8$ ie :
	 \[1s^2 2s^2 2p^6 3s^2 3p^6 4s^2 3d^{10} 4p^6 5s^2 4d^8\] Z=46 ( en réalité, c'est une exception: fini en $4p^6 4d^{10}$)
	\Question $Rb(Z=37)$ : $1s^2 2s^2 2p^6 3s^2 3p^6 4s^2 3d^{10} 4p^6 5s^1$ il y a un unique électron de valence (alcalin) on a donc souvent l'ion $Rb^+$
	\Question $Ba(Z=56)$ :$ 1s^2 2s^2 2p^6 3s^2 3p^6 4s^2 3d^{10} 4p^6 5s^2 4d^{10} 5p^6 6s^2$ Alors l'atome suivant dans la colonne à la configuration:
	$1s^2 2s^2 2p^6 3s^2 3p^6 4s^2 3d^{10} 4p^6 5s^2 4d^{10} 5p^6 6s^2 4f^{14} 5d^{10}6p^6 7s^2$ soit z$Z=88$ (radium)
	\Question On applique la règle de Klechkowski jusqu'a rencontrer la première orbitale $g$ (à savoir $5g$ car g correspond à $l=4 et 0\leq l\leq n-1$) on obtient la configuration :
	\[1s^2 2s^2 2p^6 3s^2 3p^6 4s^2 3d^{10} 4p^6 5s^2 4d^{10} 5p^6 6s^2 4f^{14} 5d^{10}6p^6 7s^2 5f^{14} 6d^{10} 7p^6 8s^2 5g^1\] 
	soit $Z=121$.
	On aurait $-4\leq m_l \leq +4$ soit 9 OA de type g possible , qui accueille donc 18 électrons (18 colonne) il serait entre les bloc $s$ et $f$.Les noyaux atomique de ce tableau serait très instable (max en labo Z=118)

\end{Answer}
