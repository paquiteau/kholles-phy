\begin{Exercise}[title=Or et Mercure]
	Le numéro atomique de l’or $(Au)$ est 79, celui du mercure
	$(Hg)$ est 80.
	\Question Donner l’état physique de ces éléments à température
	ambiante et sous une pression de 1 bar.
	\Question Quel élément de l’or ou du mercure possède la plus forte
	énergie de première ionisation ?
	\Question Même question concernant la deuxième énergie d’ionisation.
	\Question La molécule de chlorure mercurique a pour formule
	$HgCl_2$ . Justifier cette association. Quelle est la structure électronique de l’ion Hg 2+ ?
	\Question. Quelle est la structure électronique de l’ion Hg + ? Quel	atome simple possède la même structure électronique ?	Déduire de cette analogie une justification de la dimérisation de l’ion mercureux en $Hg^{2+}_2$
\end{Exercise}
\begin{Answer}
	\Question L’or est solide bien entendu et le mercure est un liquide dans les conditions imposées par l’énoncé.
	\Question Établissons les structures électroniques :$ Au (Z = 79) : [Xe]6s^1 4f^14 5d^10$
	$Hg (Z = 80) : [Xe]6s^2 4f^14 5d^10$
	La structure en 6s 1 est plus facile à ioniser que celle en 6s 2 (sous couche saturée) donc, nous aurons :
	$EI (Au) < EI (Hg)$
	\Question Pour la deuxième ionisation, le problème est inversé : on compare $6s^1$ du mercure au $5d^10$ (sous couche pleine) de l’or, donc $EI(Au^+) > EI (Hg^+)$
	\Question $Hg^{2+} (Z = 80) : [Xe] 4f^14 5d^10$ . Le mercure est réducteur et le chlore oxydant, la neutralité électrique impose la formule $HgCl_2$ .
	\Question $Hg + (Z = 80) : [Xe]6s 1 4 f 14 5d 10$. Structure de valence identique à l’hydrogène. La forme stable de l’hydrogène est le di	hydrogène $H_2$ , $Hg +$ conduit donc à l’espèce $Hg^{2+}_{2}$ stable.
\end{Answer}
