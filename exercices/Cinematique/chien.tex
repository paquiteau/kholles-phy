\begin{Exercise}[title=La courbe du chien]
  Il s’agit d’établir l'équation de la trajectoire que parcourt un chien lorsque son maître, éloigné, l'appelle tout en se déplaçant sur une droite perpendiculairement au segment de droite les séparant à l’instant origine. On conviendra qu'à cet instant, le maître est à une distance $a$ du chien ; il se déplace de façon uniforme à la vitesse $v$. le chien accourt à la même vitesse $v$ et de façon à être, à tout moment, dirigé vers son maître. On peut donc choisir un repère orthonormé, tel que le chien soit au point $ O$ de coordonnées $(0,0)$ et le maître au point $A$ de coordonnées $(a,0)$ au départ, la droite décrite par le maître étant celle d’équation $x = a$. On désigne par $x$ et $y$ les coordonnées du chien au bout du temps $t$.
  \Question Faire un schéma
  \Question Déterminer l'équation de la courbe.
\end{Exercise}
\begin{Answer}
  Au bout du temps « $t$ », le maître est en  « $M$ », tel que $AM = vt$.
  Désignons par « $l$ » l'arc de trajectoire « $OC$ » parcouru par le chien à la vitesse $v$. Calculons la pente au point « $C$ »  :
\[\frac{\d y}{\d x} = \tan(\alpha)= \frac{BM}{BC} =\frac{\d y}{\d x} = \frac{vt-y}{a-x} \]
D'autre part, au temps $t$, le chien a parcouru une longueur $l = vt$, ce qui donne :
\[(a-x)\frac{\d y}{\d x} +y = l \]
avec $x \leqslant a.$En dérivant par rapport à $x$, on obtient :
\[\left[(a-x)\frac{\d y}{\d x}\right]' = -\frac{\d y}{\d x} + (a-x)\frac{{\d^2} y}{\d x^2}\]
Soit :
\[-\frac{\d y}{\d x} + (a-x)\frac{{\d^2} y}{\d x^2} +  \frac{\d y}{\d x} = \frac{\d l}{\d x} \implies
(a-x)\frac{{\d^2} y}{\d x^2} = \frac{\d l}{\d x}\]
En utilisant le fait que $\d l^2 = \d x^2+\d y^2 $, et en posant $y'=\frac{\d y}{\d x} $, on obtient finalement  l'équation suivante:
\[(a-x)\frac{\d y'}{\d x} = \sqrt{1+y'^2}
\implies \frac{\d y'}{\sqrt{1+y'^2}} = \frac{\d x}{a-x} \]
En intégrant on obtient :
\[\ln (y'+ \sqrt{1+y'^2}) = -\ln(a-x)+C \]
À l'instant $t=0$, le chien est au point O et sa trajectoire est tangente à la droite des abscisses, donc $y'=0$ et $x=0$, ce qui donne pour la constante : $ C = \ln(a)$
Il vient alors :
\[\ln (y'+ \sqrt{1+y'^2}) = -\ln(a-x) + \ln(a) = \ln\left(\frac{a}{a-x}\right) \]

ou encore :
\[y'+ \sqrt{1+y'^2} = \frac{a}{a-x}\]
De l'équation aux logarithmes ci-dessus, on déduit aussi l’égalité des inverses, et en utilisant la simplification de la racines au carré  :
\[
  \frac{1}{y'+\sqrt{1+y'^2}} = \frac{a-x}{a} \implies
  y' - \sqrt{1+y'^2} = -\frac{a-x}{a}
\]

puis, en faisant la somme des deux dernières équations il vient :
\[y'- \sqrt{1+y'^2} + y'+\sqrt{1+y'^2} = -\frac{a-x}{a} + \frac{a}{a-x}\]
L'équation donnant $y'$ est alors :

\[y' = \frac{1}{2} (-\frac{a-x}{a}+\frac{a}{a-x})\]
On peut alors intégrer :
\[ y = \frac{1}{2} \left(\int{-\frac{a-x}{a}}\d x + \int{\frac{a}{a-x}}\d x\right).\]
Si k= 1 \[y=\frac{(a-x)^2}{4a}-\frac{a}{2}\ln (a-x) +D\]
 et on détermine comme auparavant la constante D à partir de la valeur initiale $x=0, y=0$,  qui fournit : $D=\frac{a}{2}\ln a-\frac{a}{4}$
\[y=\frac{x^2-2ax}{4a}+\frac{a}{2}\ln{\frac{a}{a-x}}\]

\end{Answer}
