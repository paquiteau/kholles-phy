\begin{Exercise}[title=Satellite Géostationnaire]
	\Question Déterminer l'altitude des satellites géostationnaires.\\
	\emph{Donnée : $R_T=6400 km$ $M_T=$\SI{5,972e24}{\kg} $\mathcal{G}=$\SI{6.67e-11}{\N\m^{2}\kg^{-2}}}
\end{Exercise}
\begin{Answer}
  PFD sur le satellite dans un mouvement circulaire uniforme ($a =-V^2/(h+R_T) = (h+R_T)\Omega^2$)avec $\Omega = 2\pi/T$ et $T=24h=86400s$:
  \[
    M_s \frac{V^2}{(h+R_T)} = -\mathcal{G}\frac{M_sM_T}{R_T+h} \implies h = \sqrt[3]{\frac{\mathcal{G}M_TT^2}{4\pi^2}} -R_T \equals_{A.N.} \text{\SI{36e3}{km}}
  \]
\end{Answer}
