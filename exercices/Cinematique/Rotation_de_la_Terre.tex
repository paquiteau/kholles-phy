\begin{Exercise}[title=Rotation de la Terre]
	\Question On suppose que le mouvement de la Terre autour du Soleil est suivi avec une vitesse angulaire $\dot{\theta}$ constante. Déterminez sa vitesse angulaire de rotation
	\Question L'étude dynamique montre que la quantité
	$r^2\theta$ est	constante au cours du mouvement de la Terre, où r est la distance de la Terre au Soleil (loi des aires).
	 En déduire que l'orbite de la Terre est circulaire et déterminez sa vitesse.
	\Question Exprimez la position, la vitesse et l'accélération de la Terre au cours du temps en coordonnées polaire et
	cartésienne.
\end{Exercise}
\begin{Answer}
	Distance Terre Soleil : \SI{150e6}{km}
\end{Answer}
