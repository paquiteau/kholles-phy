\begin{Exercise}[title=Le lièvre et le camion]
	Un lièvre essaie de traverser une route, de largeur $L$ sur laquelle roule un camion. Initialement, le lapin est à une distance $d_0$
	du camion. Il court à la vitesse $v_l$ 	en formant un angle $\theta$ avec l'horizontal, tandis que le camion roule à la vitesse $v_c$,
	\Question Déterminez le temps nécessaire au lapin pour traverser la route.
	\Question Exprimez la distance entre le camion et le lapin au cours du temps. A quelle condition sur cette distance le
	lapin reste-t-il entier ?
	\Question Déterminez la valeur minimale de la vitesse	pour laquelle le lapin peut traverser la route sain et sauf.
	\Question (*) Comment s'appelle la femelle du lièvre ?
\end{Exercise}
\begin{Answer}
	\Question Pas de frottement , problème plan
	\Question Résolution en cartésien! $\begin{cases}
	x= v_0 \sin \alpha t \\
	y= \frac{-gt^2}{2} +v_0 \cos \alpha t + h_0
	\end{cases}$
	\Question $x^2 + \left(y-\left(h_0-\frac{gt^2}{2}\right)\right)^2 = v_0^2 t^2$ équation de cercle , par symétrie on a une sphère.
\end{Answer}
