\begin{Exercise}[title=La pomme]
	On se place juste à la surface d'une planète de rayon R et de champ gravitationnel g .Isaac jette  une	pomme avec une vitesse $v_0$ purement horizontale	on négligera les frottements de l'air.
	\Question En supposant la planète localement plane, déterminez la hauteur $dz$ dont est tombée la	pomme après avoir parcouru une longueur dx .
	\Question Après une distance horizontale dx, de combien le sol de la planète est il descendu ? On
	pourra utiliser les formules pour les petits
	angles $\tan \theta\simeq \theta , \cos \theta \simeq  1-\theta^2$.
	\Question En déduire qu'il existe une certaine vitesse $v_0$
	pour laquelle la pomme va revenir à son point
	de départ.
\end{Exercise}
\begin{Answer}
\Question chute libre autour de 0
\Question $dh=(1-\cos d\theta)R=\frac{dx^2}{2R}$
\Question $v_0=\sqrt{gR}$
\end{Answer}
