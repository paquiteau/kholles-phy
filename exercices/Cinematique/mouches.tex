\begin{Exercise}[title=Quatre mouches et un carré]
  Quatre mouches supposées ponctuelles sont initialement situées aux sommets $A, B, C$ et $D$ d'un carré d'arête $a$. Chaque mouche vole en direction de la suivante à la même vitesse $v$ en norme. En tenant compte de la symétrie du problème, on peut dire qu'à chaque instant les mouches occupent les $4$ sommets d'un carré dont l'arête tend vers $0$. On note $O$ le centre commun de tous ces carrés.
  \Question Exprimer la vitesse de la mouche $A$ dans le repère polaire de deux façons différentes.
  \Question Sachant que les mouches se rencontreront au centre du carré $ABCD$, déterminer le temps qu'elles vont mettre pour se rencontrer.
  \Question En utilisant la question $(a)$, donner l'équation polaire de leur trajectoire (appelée spirale logarithmique).
\end{Exercise}
\begin{Answer}

\end{Answer}

%%% Local Variables:
%%% mode: latex
%%% TeX-master: "nil"
%%% End:
