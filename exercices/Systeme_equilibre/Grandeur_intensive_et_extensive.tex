\begin{Exercise}[title=Grandeur intensive et extensive]
		Soit une mole d’un gaz occupant une volume $V_m$ sous la pression $P$ et à la température $T$.
		\Question On suppose que ces grandeurs sont liées par l'équation $\left(P+\frac{a}{V^2}\right)(V_m-b) =RT$ où $a,b,R$ sont des constantes. utiliser les propriétés d'intensivité ou d'extensitivé des grandeurs pour établir l'équation correspondante relative à $n$ moles.
		\Question Même question pour l'équation: $P(V_m-b)\exp\left(\frac{a}{RTV_m}\right)$
\end{Exercise}
\begin{Answer}
  \Question Comme $V_m = \frac{V}{n}$ on a,
  \[\left(P+\frac{a}{V_m^2}\right)(V_m-b)=RT \equiv \left(P+\frac{n^2a}{V^2}\right)\left(\frac{V}{n}-b\right) = RT \equiv\left(P+\frac{n^2a}{V^2}\right)\left(\frac{V}-bn\right) = nRT\]
  $B=nb$ est extensive car additive et $A=n^2a$ aussi (dépend de la quantité)
  \Question $P(V-nb)\exp\left(\frac{na}{RTV}\right)=nRT$
\end{Answer}
