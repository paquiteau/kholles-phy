\begin{Exercise}[title=Densité particulaire et volume molaire]
		\Question calculer le nombre de molécule par $cm^3$ dans un gaz parfait à \SI{27}{\degreeCelsius} sous une pression de $10^{-6}$ atmosphère.
		\Question Calculer le volume occupé par une mole d'un gaz parfait à la température de \SI{0}{\degreeCelsius} sous la pression atmosphérique normale. En déduire l'ordre de grandeur de la distance moyenne entre molécule.

\end{Exercise}
\begin{Answer}
		\Question D'après l'équation d'état du gaz parfait le nombre de molécules par unité de volume est $n^*=\frac{N}{V} =\frac{P}{k_BT} \simeq$\SI{2,5e19}{\per\m^3}
		\Question le volume molaire cherché est $V_m =\frac{RT}{V}=$\SI{22,4}{L}
\end{Answer}
