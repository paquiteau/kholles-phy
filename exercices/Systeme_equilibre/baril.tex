\begin{Exercise}[title=Baril écrasé]
  Pour effectuer une démonstration de physique, un groupe d’étudiants porte de
  l’eau à ébullition, à pression ambiante, dans un ancien baril de pétrole
  (contenance \SI{208}{\liter}, hauteur \SI{88}{\centi\metre}).

  Le baril est retiré de la source de chaleur et fermé de façon hermétique. Le
  but de l’opération est de pouvoir observer le baril se faire écraser par
  l’atmosphère suite au changement d’état de l’eau qu’il
  \Question Quelle dépression peut-on générer à l’intérieur du baril en le
  laissant se refroidir ?

  \Question Quelle serait alors la force verticale s’appliquant sur la paroi
  supérieure du baril ?

  \Question Il reste \SI{5}{\liter} de liquide au fond du baril à la fermeture
  du bouchon. Quel est le titre de la vapeur ?
  \Question Quelle masse de vapeur s’est condensée pendant le refroidissement ?
  \Question Combien a-t-il fallu retirer de chaleur pour atteindre la dépression finale ?
\end{Exercise}
\begin{Answer}

  \Question  Si on atteint $T_\B = \SI{30}{\degreeCelsius}$ à volume constant, alors
  $p_\text{intérieur min.} = p_{\text{sat.} \SI{30}{\degreeCelsius}} =
  \SI{0,004247}{\mega\pascal}$. Alors $\Delta p_\text{max} = \SI{-9,575e4}{\pascal}$ ;

  \Question  $F_\text{max} = \Delta p_\text{max}
S_\text{couvercle} = \SI{22,6}{\kilo\newton}$ (le baril sera bien sûr écrasé avant)
  \Question  $x_\A = \num{0,02408}$, $x_\B = \num{7,328e-4}$ ;
  \Question  $m_\text{condensée} = \SI{4,9138}{\kilogram}$ ;
  \Question  ${Q}_\fromatob = \SI{-1,878}{\mega\joule}$.
\end{Answer}
