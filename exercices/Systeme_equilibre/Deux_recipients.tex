\begin{Exercise}[title=Deux récipients]
		Un récipient $(A)$ de volume $V_A= 1L $ contient de l'air à $t_A=$\SI{15}{\degreeCelsius} sous une pression $P_A=72cmHg$.\\
		Un autre récipient $(B)$ de volume $V_B=1L$ contient également de l'air à $t_B=$\SI{20}{\degreeCelsius} sous une pression $P_B=45 atm$
		On réunit $(A)$ et $(B)$ par un tuyau de volume négligeable et on laisse l'équilibre se réaliser à $t=$ \SI{15}{\degreeCelsius}\\
		On modélise l'air par un gaz parfait de masse molaire $M=$\SI{29}{\g\per\mol}.
		\emph{Donnée : le centimètre de mercure est défini par ma relation $1 atm = 76cmHg$}
		\Question Quelle est la pression finale de l'air dans les récipients ?
		\Question Quelle est la masse d'air qui a été transféré d'un récipient à un autre ?
\end{Exercise}
\begin{Answer}
		\emph{Indication: Exprimer initialement les quantités de matières totale. L'état final étant un état d'équilibre thermodynamique, les variables intensives sont uniformes, dont la densité moléculaire et la pression . En déduire les quantités de matières finales.}
		\Question $m_{B\to A}=26.1g$
		\Question $P=22.5 bar$
\end{Answer}
