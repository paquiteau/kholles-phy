\begin{Exercise}[title=Cube flottant]
  Un cube de côté a, de masse volumique $\rho_c$ , flotte en équilibre dans un
  liquide de masse volumique $\rho_L$ $(\rho_L > \rho_c)$. Les conditions sont
  telles que le cube ne bascule pas, gardant toujours sa face inférieure
  horizontale. On ne prend pas en compte la pression de l’air, ni les
  frottements visqueux avec le fluide. On choisit un repère dont l’origine se
  situe au niveau de la base du cube lorsqu’il est à l’équilibre dans le fluide.
  A l’instant $t = 0$ on enfonce le cube dans le fluide (hauteur de cube
  immergée $h_0$) et on le lâche sans vitesse initiale.
  \Question Établir l’équation différentielle du mouvement dans le cas où $h_0<a$.
  \Question En déduire l’équation du mouvement du cube en fonction du temps. Préciser la
  période des oscillations
  \Question Quelle doit être la valeur de $h_0$ pour
  que le cube puisse bondir hors du liquide ?
\end{Exercise}
\begin{Answer}
  \Question Référentiel: terrestre supposé galiléen.\\
  Système : le cube de masse constante $m = \rho ca^3$ \\
  Repère : R(O,$\vec{k}$) avec O à la base du cube à l’équilibre et $\vec{k}$
  dirigé vers le bas.\\
  Mouvement : mouvement rectiligne vertical. \\
  Système de coordonnées cartésiennes, position de la base du cube  $z=zbase$.\\
  Cinématique: $\vec{OM}= z\vec{k} ; \vec{v} = \dot{z}\vec{k} ; \vec{a}=
  \ddot{z}\vec{k}$\\
  Forces extérieures :\\
  Forces à distance : poids du cube
  $\vec{P} = m_{cube} g\vec{k} = \rho ca^3 g \vec{k}$ \\
  Forces de contact : poussée d’Archimède $\vec{F} = -m_l g \vec{k} -\rho_l a^2(h_{eq} + z)g\vec{k}$
  où$h_{eq}$ est la hauteur immergée à l’équilibre. \\
  PFD Oz:$ \rho ca^3g -\rho_L(h_{eq} + z)g = \rho_ca\ddot{z}$ La situation d’équilibre
  telle que $\ddot{z}= 0 $ et $z = 0$ donne la hauteur immergée à l’équilibre
  $h_{eq} =\frac{\rho_c}{\rho_l}a$ L’équation différentielle du mouvement est
  donc :$\ddot{z}+\frac{g\rho_L}{a\rho_c}z=0$
  \Question \emph{attention aux CI}
  $z= (h_0-h_{eq})cos(\omega t)$ et $T=2\pi\sqrt{\frac{a\rho_c}{g\rho_L}}$
  \Question Le cube bondit hors de l’eau si la base du cube atteint la surface
  de l’eau, c’est à dire si la valeur minimale de $z$ vaut $-h_{eq}$ ce qui
  s’écrit $-(h_0 -h_{eq} ) = -h_{eq} $, d’où la condition $h_0 > 2h_{eq}$ pour
  bondir hors de l’eau. Avec la condition$ h_0 < a$, ce résultat implique que
  $h_{eq} < a/2$, c’est à dire $\rho_c < \rho_L/2$. La masse volumique du cube
  doit donc être inférieure à la moitié de celle du liquide pour que cette
  situation soit possible.
\end{Answer}
