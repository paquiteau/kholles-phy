\begin{Exercise}[title=Suite de ressort]
  Un solide S, de masse m, est accroché au plafond par l’intermédiaire d’un
  ressort $R_1$ de masse négligeable de raideur $k$ et de longueur à vide $l_0$.
  Un second ressort $R_2$, identique au premier, pend sous le solide . À partir
  de l’instant $t = 0$ on tire sur le ressort $R_2$ avec une force . On constate
  que, si l’on accroît très lentement $F$ , l’un des ressorts finit par se
  briser et que, si l’on accroît très rapidement $ F$, c’est l’autre ressort qui
  se brise.
  \Question Expliquer quel est, dans chacun de ces deux cas, le ressort
  qui se brise.
  \Question La force appliquée à l’extrémité libre de R2 varie
  avec l’instant t positif selon la loi $F=m\alpha t$ où $\alpha$ est une
  constante positive. La tension T de chaque ressort suit la loi de \textsc{Hooke},
  jusqu’à une tension de rupture $T_r$ où x est l’allongement du ressort par
  rapport à sa longueur à vide. On pose $\omega:\sqrt{\frac{k}{m}}$ et on
  appelle x l'allongement de $R_1$. A l'instant 0 le système est encore à
  l'équilibre. Exprimer $x(t)$
  \Question Exprimer $T_2-T_1$ en fonction de $t$
  \Question Discuter selon la valeur de $\alpha$ le ressort qui casse en premier.
\end{Exercise}
\begin{Answer}
  \Question lent : $R_1$ rapide $R_2$
  \Question - on isole $\{R_2\}$ : $\vec{T_2} = -\vec{F}$ ( ressort sans masse) \\
  - on isole $\{m\}$ :
  \[ m\ddot{x} = mg \underbrace{-kx}_{T_1} + m\alpha t\]
  Après résolution (solution générale et particulière et détermination
  constantes) :
  \[x= \frac{mg}{k}(1-\cos(\omega t) +\frac{m\alpha}{k}\left(t-\frac{\sin(\omega
        t)}{\omega}\right)\]
  \Question
  $ T_2-T_1 =\frac{m\alpha}{\omega}\sin(\omega t) -mg$
  \Question si $\alpha < g\omega$ c'est $R_1$ qui casse si $\alpha > g\omega$l'un ou l'autre. $\alpha \gg
  g\omega$ c'est$ R_2$
\end{Answer}
