\begin{Exercise}[title=Suspension de voiture]
  Une automobile de masse \gdr{m}850{}{kg} est schématisée par une carrosserie
  de masse \gdr{m_1}{700}{kg} reposant par l'intermédiaire de quatre ressorts de
  raideur \gdr{k}{6950}{\N\per\m} sur quatre roues, chacune de masse $m_2$=
  \SI{37.5}{\kg}
  \Question Calculer la hauteur dont il faut soulever l  carrosserie pour que
  les roues décollent du sol.
   \Question Calculer la période des oscillations verticales de la carrosserie.
  \Question Pourquoi cela n'arrive pas en réalité?
\end{Exercise}
\begin{Answer}
  \Question En charge la suspension est raccourci de $x_0=\frac{m_1g}{4k}$. Quand
  la carrosserie est soulevée et que les roue décolle du sol la suspension est
  étirée de$x'=\frac{m_2g}{k}$ il faut donc soulever la carrosserie de
  $x=x_0+x'=0.30m$
  \Question
  \[m_1\ddot{x} =-m_1g -4kx \Rightarrow \omega^2 =\frac{4k}{m_1}\text{ et }
    T=\pi\sqrt{\frac{m_1}{k}} = 1,0 s\]
  \Question présence d'amortisseur
\end{Answer}
