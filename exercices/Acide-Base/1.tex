\begin{Exercise}[title=]
	On considère  les couples acido-basique suivants dont on donne les $pKa$
	\begin{center}
		\begin{tabular}{c|c|c|c|c|}
		 &$HCOOH/HCCOO^-$ & $HClO/ClO^-$ & $HSO_4^-/SO_4^{2-}$ &$HBO_2/BO_2^-$\\
		 \hline
		 pKa & 3,7 & 7,5 & 1,9 & 9,2
	\end{tabular}
	\end{center}
\Question Tracer un diagramme de prédominance de ces différentes espèces.
\Question Déterminer la réaction ayant la constante thermodynamique la plus grande dans le cas des mélanges suivants obtenus dans 1 Litre de solution aqueuse et déterminer sa constante:
\subQuestion 1 mole de méthanoate de sodium, 2 moles d'acide hypochloreux ($HClO$), 1mole de sulfate de sodium.
\subQuestion 1 mole de borate de sodium, 2 moles d'hypochlorite de sodium ($NaClO$), 1mole de méthanoate de sodium.
\subQuestion 1 mole d'hydrogénosulfate de sodium, 2 moles de borate de sodium, 1mole de soude.

\end{Exercise}
\begin{Answer}
	\Question
\begin{center}
\begin{tabular}{cccccccccccc}
 	$H_3O^+$&\vline &\multicolumn{9}{c}{$H_2O$}	 								 \\ \hline
	\multicolumn{3}{r}{ $HSO_4^-$ }	&\vline & \multicolumn{7}{l}{$SO_4^{2-}$}	 \\ \hline
	\multicolumn{5}{r}{$HCO_2H$}	&\vline	& \multicolumn{5}{l}{$HCO_2^-$} 	 \\ \hline
	\multicolumn{7}{r}{$HClO$} 		&\vline &\multicolumn{3}{l}{$ClO^-$}		 \\ \hline
	\multicolumn{9}{r}{$HBO_2$}		&\vline &\multicolumn{2}{l}{$BO_2^-$}		 \\ \hline
		&	0	&			& 1,9	&			&	3,7	&			&	7,5	&		&	9,2	&		 \\
\end{tabular}
\end{center}
\Question Les trois espèces appartienne à des domaines communs, àl'équilibre elles constituent les espèces majoritaires. Pour le premier mélange la RP est : $HClO+HCOO^- =ClO^-+HCOOH$ de constante \gdr{K^o}{e-3,8}{}.
\Question Les trois espèces appartienne à des domaines communs, àl'équilibre elles constituent les espèces majoritaires. Pour le second mélange la RP est : $BO_2^-+H_2O = HBO_2+HO^-$ de constante \gdr{K^o}{e-4,8}{}.
\Question  pour le dernier mélange l'acide le plus fort est $HSO_4^{-}$ et la base la plus forte $HO^-$ d'où la RQ :
$HSO_4^{-}+HO^- =SO_4^{2-}+H_2O$ de constante \gdr{K^o}{e12,1}{}.

\end{Answer}
