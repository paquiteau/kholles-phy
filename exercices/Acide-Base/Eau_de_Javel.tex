\begin{Exercise}[title=Eau de Javel]
	Louis Berthollet (1748-1822),chimiste français, médecin de formation mis au point l'eau de Javel pour les lavandières des bords de Seine à Javel, petit village aux portes de Paris à l'époque.
	\Question L'eau de Javel est obtenue par dissolution du dichlore gazeux dans une solution de soude. Écrire le bilan de cette réaction, et calculer sa constante
	\Question Pourquoi ne faut-il jamais mélanger de l'eau de Javel avec un détartrant lorsqu'on fait le ménage? Indiquer la réaction ayant lieu ainsi que sa constante.

	Lors de la première guerre mondiale, le chimiste Henry Dakin mit au point un antiseptique (dont la substance active est l'eau de Javel) pour les plaies ouvertes ou infectée, dans le cadre des travaux de ce dernier sur le traitement des plaies de guerre.
	La solution est à base d'eau de Javel additionnée de permanganate de potassium pour la stabiliser.
	\Question Une mesure de l'absorbance de la solution de Dakin ( attribuée uniquement à $KMnO_4$)indique à 525 nm : $A =0,15$ (pour une cuve de largeur $1cm$). En déduire le pourcentage massique de $KMnO_4$ dans la solution.
	\Question Le degré chlorométrique $D$ d'une eau de Javel est le volume de dichlore maximal libéré lors de la réaction d'acidification d'un litre de la solution (à \SI{0}{\degreeCelsius}) sous 1 atm) Sachant que l'eau de Dakin indique $0,500g$ d'hypochlorite de sodium pour 100 mL, déterminer son degré chlorométrique.

	\emph{Données:}
	\begin{itemize}
		\item \gdr{M(K)}{39.1}{\g\per\mole} \gdr{M(Mn)}{54,9}{\g\per\mole}, \gdr{M(Cl)}{35,5}{\g\per\mole}
		\gdr{M(Na)}{23,0}{\g\per\mole}
		\item $E^o(Cl_{2(g)}/Cl^-) =1,36 $V ; $E^o(HCl0_{aq}/Cl_{2(g)}) =1,63 $V ;
		$pKa(HClO/ClO^-) = 7,2$
		Coefficient d'absorption molaire de $KMnO4 $ à 525 nm \gdr{\varepsilon_{525nm}}{2300}{\L\per\mole\cm}
	\end{itemize}

\end{Exercise}
\begin{Answer}
	\Question dismutation du dichlore en milieu basique : $Cl_{2(g)}+2HO^-\rightleftharpoons Cl^-+ClO^-+H_2O$ $K=10^{16,3}$
	\Question les détartrant sont acides , on favorise la médiamutation et on a un dégagement de $Cl_2$ , très toxique. $Cl^-+HClO +H^+ \to Cl_{2(g)}+H_2O$ $K=10^{+4,5}$.
	\Question $[MnO_4^-] = A/\epsilon l = $\gdr{}{65,2}{\micro\mole\per\L} donc pour 1L on a \gdr{m_{KMnO_4}}{10,3}{mg}
	\Question $n_{Cl_2}=n_{ClONa}$ donc on a  un degrée chlorométrique de \gdr{}{1,5}{\degree}

\end{Answer}
