\begin{Exercise}[title=]
	Déterminer les concentrations à l'équilibre pour une solution d'hydrogénocarbonate de sodium $NaHCO_3$ (couples $AH_2/AH^-$ , $pKa_1=6,4$ et $AH^-/A^{2-}$, $pKa_2=10,4$) de concentration \gdr{C}{0,1}{\mole\per\L}. En déduire le $pH$.

\end{Exercise}
\begin{Answer}
	La RP est la réaction d'amphotérisation de constante $K^o=K_{A2}/K_{A1} =10^{-4}<1$
	\begin{center}
		\begin{tabular}{cccccc}
		&$2HA^-$&$\rightleftharpoons$& $A^{2-}$ & +& $H_2A$ \\
		 \hline
	 EI & $C$  	& 					 &	0		&  &0\\
		 \hline
	 EF & $C-2x$&					 & x 		&  &x\\
	\end{tabular}
	\end{center}
Il vient pour une RP peu avancée : $x \simeq C \sqrt{K} =10^{-3} \ll C$.
le pH peux se calculer à partir des deux couples acido-basique:
\[pH = pK_{A1}+\log\frac{[HA^-]}{[H_2A]}\simeq 8,4 \]
Valide car $h, \omega \ll x$ (AN!) .

\end{Answer}
