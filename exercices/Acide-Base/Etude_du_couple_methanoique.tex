\begin{Exercise}[title=Étude du couple méthanoique]
	\emph{Données:}
	\begin{itemize}
		\item Densité de l'acide acétique $d=1,05$
		\item Masse molaire de l'acide acétique $c_m=$\SI{60}{\g\per\mol}
		\item $pKa CH_3COOH/CH_3COO^- = $ 4,8
	\end{itemize}
	\Question Tracer le diagramme de prédominance de l'acide acétique et de l'ion acétate en solution aqueuse.
	\Question Tracer  également l'allure du diagramme de répartition en solution aqueuse.
	\Question On constitue une solution aquese de la manière suivante: dans une fiole jaugée de $V_0=500mL$ est introduit un volume $V_1=10mL$ d'acide acétique pur. On complète au trait de jauge avec de l'eau distillée. Une analyse rapid0 du pH le situe entre 2 et 3.
	\subQuestion Déterminer la concentration apportée en acide acétique dans la solution.
	\subQuestion Écrire l'équation chimique de mise en solution aqueuse de l'acide acétique.
	\subQuestion En observant le diagramme de répartition que peut on déduire du résultat fourni par le papier pH?
	\subQuestion En déduire par le calcul le plus simple possible, la concentration de toutes les espèces en solutions et donner la valeur du pH avec assez de chiffre significatif.
	\Question À la solution précédente est ajouté un volume $V_b=$ 100mL d'une solution de soude de concentration $C_b=1,00$ \si{\mol\per\L}.
	\subQuestion Quelle est la nouvelle concentration apportée d'acide acétique dans la solution?
	\subQuestion Quelle est la nouvelle concentration apportée d'hydroxyde de sodium?
	\subQuestion Écrire l'équation chimique de la réaction acido-basique entre la soude et l'acide acétique. Calculer sa constante d'équilibre; conclure. Faire un bilan de concentration en ne considérant que cette réaction.
	\subQuestion Quelles sont les espèces majoritaires et minoritaires dans cette solution ? Justifier la réponse qualitativement, puis numériquement.
	\subQuestion Quelles sont les propriétés de la solution obtenue à l'équilibre?
\end{Exercise}
\begin{Answer}
	\Question $CH_3COOH ~ ~|_{4,8} ~ ~CH_3COO^-$
	\Question Courbe qui monte et descende, intersection au pKa.
	\Question
	\subQuestion $[CH_3COOH]=\frac{n}{V}=\frac{\rho V_1}{M} =$ \SI{0.35}{\mol\per\L}
	\subQuestion $CH_3COOH + H_2O = CH_3COO^- + H_3O^+$ faire le tableau d'avancement
	\subQuestion On en déduit $h<< C_0$
	\subQuestion $K_a =\frac{h^2}{C_0-h} \simeq \frac{h^2}{C_0}$ Alors $pH=2,6$
	\Question
	\subQuestion \subQuestion  $C'_{AH}$ = \SI{0.29}{\mol\per\L}. et $C_{OH}=$\SI{0.17}{\mol\per\L}
	\subQuestion $CH_3COOH + H_O^- = CH_3COO^- + H_2O$ faire le tableau d'avancement.
	\subQuestion $pH= pKA + \log(\frac{A^-}{AH}) \simeq 4,9$.
	\subQuestion les ions hydrique sont minoritaires.
	\subQuestion Solution tampons.

\end{Answer}
