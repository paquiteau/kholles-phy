\begin{Exercise}[title=Courant équivalent]
  Calculer le courant circulant dans $R_1$:
  \begin{center}
    \begin{circuitikz}
      \draw (0,-1) to[V,v=$E_1$,] (0,1) to[R, label=$r$](0,3) -- (1,3)
      to[R,label=$R_1$](3,3) -- (4,3) to[R,label=$r$](4,1)
      to[V,v<=$E_2$](4,-1) -- (0,-1); \draw (1,3) to[R,
      label=$R_2$](1,-1); \draw (3,3) to[R, label=$R_2$](3,-1);
    \end{circuitikz}
  \end{center}
\end{Exercise}
\begin{Answer}
  Solution Théorème de superposition : on traite d'abord un générateur
  puis l'autre. Attention aux signes : E1 donne une intensité dans un
  sens, E2 dans l'autre.  En faisant thevenin - norton, on trouve que
  E1 crée $i_1 =\frac{R_2}{2rR_2+R_1(R_2+r)}E_1$ donc l'intensité
  totale est $i = =\frac{R_2}{2rR_2+R_1(R_2+r)}(E_1-E2)$
\end{Answer}
