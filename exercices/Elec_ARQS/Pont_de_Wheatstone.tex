\begin{Exercise}[title=Pont de Wheatstone]
  Un pont de Wheatstone est un montage électrique permettant de déterminer une   résistance inconnue.La résistance à déterminer est$R_1$. $R_3$ et $R_4$ sont fixe et connue. $R_2$ est une résistance variable dont on connait la valeur.\\
  Le pont est dit équilibré si $u = 0$.
  \begin{center}
    \begin{circuitikz}[scale=0.6]
      \draw (0,-2) to[short,-*] (0,3)node[left]{A} to[R,l=$R_1$] (3,6)
      to[R,l=$R_2$,-*](6,3)node[right]{B} -- (6,-2) to[V,l=$E$] (0,-2);
      \draw (0,3) to[R,l=$R_3$] (3,0) to[R,l=$R_4$] (6,3); \draw (3,0)
      to[open ,v=$u$] (3,6);
    \end{circuitikz}
  \end{center}
  \Question Déterminer $u$ en fonction de E et des résistances.
  \Question À quelle condition le pont est il équilibré? Déterminer alors $R_1$.\\
  \emph{Donnée: $R_3$=\SI{100}{\ohm}; $R_4$=\SI{5}{k\ohm};
    $R_2$=\SI{1827}{\ohm}; E= \SI{6}{V}} \Question Le votmètre indique
  la tension $u=0$ si on a $|u|< $ \SI{1}{mV}. Dans le cadre de
  l'application numérique de 2) , donner la précision de la mesure de
  $R_1$.
\end{Exercise}
\begin{Answer}
  \Question $u=\frac{R_3}{R_3+R_4}-\frac{R_1}{R_1+R_2}$ \Question
  $R_1=$\SI{36.5 +- 0.3}{\ohm}
\end{Answer}
