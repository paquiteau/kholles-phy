\begin{Exercise}[title=Pont de Maxwell]
  On considère à présent le montage suivant, dans lequel la valeur
  d'impédance de la bobine et la résistance sont inconnues. On peut
  faire librement varier $R_1$, tandis que les autres valeurs sont
  connues. Lorsque le pont est équilibré, on a $R_1=$
  \SI{1}{\kilo\ohm} $R_2=$\SI{42}{\kilo\ohm} $R_3=$\SI{100}{\ohm} et
  $C=5nF$. Déterminer r et L.
  \begin{center}
    \begin{circuitikz}[scale=0.7]
      \ctikzset{bipoles/length=1cm} \draw (0,-2) to[short,-*] (0,3)node[left]{A}
      to[R,l=$R_1$] (3,6) --(4,5)
      --(3.5,4.5)to[R,l_=$R_2$](4.5,3.5)--(5,4)to[short,-*] (6,3)node[right]{B}
      -- (6,-2) to[V,l=$E$] (0,-2); \draw (4,5) --(4.5,5.5)
      to[C,l=$C$](5.5,4.5) --(5,4); \draw (0,3) to[R,l=$r$]
      (1.5,1.5)to[L,l=$L$](3,0) to[R,l=$R_3$] (6,3); \draw (3,0)
      to[open ,v=$u$] (3,6);
    \end{circuitikz}
  \end{center}
\end{Exercise}
\begin{Answer}
  $u = \left[
    \frac{r+jL\omega}{r+R_3+jL\omega}-\frac{R_1(1+jR_2C\omega)}{R_2+R_1(1+jR_2C\omega)}\right]E$
  alors $r= \frac{R_1R_3}{R_2}$
\end{Answer}
