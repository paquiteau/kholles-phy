\begin{Exercise}[title=Formule des opticiens]
On accole deux lentilles de vergences $V_1$ et$V_2$ de centre $O_1$ et$O_2$. l'ensemble est suffisamment mince pour pouvoir assimiler l'association à un système optique centré et mince de centre optique $O \simeq O_1 \simeq O_2$.\\
Montrer que l'association est équivalente à une lentille $L_{eq}$ dont on précisera la position du centre optique $O_{eq}$,ainsi que la vergence $V_{eq}$ et le grandissement $\gamma_{eq}$.
\end{Exercise}
\begin{Answer}
Pour un objet $AB$ transverse avec $A$ sur l'axe optique on à le système: \\
$AB \xrightarrow{L_1} A_1 B_1 \xrightarrow{L_2} A'B'$:
\[
\left\lbrace
\begin{array}{l}
\frac{1}{\overline{OA_1}}-\frac{1}{\overline{O A}} = V_1 \\
\frac{1}{\overline{OA'}}-\frac{1}{\overline{OA_1}} = V_1 \\
\end{array}\right. \Longrightarrow  \frac{1}{\overline{OA'}}-\frac{1}{\overline{OA}} = V_1+V_2= V_{eq}
\]
et pour le grandissement: $\gamma_{eq}=\frac{\overline{A'B'}}{\overline{AB}} \frac{\overline{A'B'}}{A_1 B_1}\frac{A_1 B_1}{AB}=\gamma_1 \times \gamma_2 $
\end{Answer}
