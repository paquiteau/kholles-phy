\begin{Exercise}[title=Lentille Gravitationnelle]
 \Question On considère un rayon lumineux arrivant depuis l'infini sur une
 lentille de distance focale $f_0$ . Ce rayon frappe la lentille à une distance $r_0$ du centre optique. Déterminez la déviation angulaire $\delta\phi$
du rayon(angle entre le rayon incident et le rayon émérgent de la lentille), en considérant $r_0\ll f_0$.
\Question
On observe depuis la Terre une galaxie supposée à l'infinie. Sur la ligne de
visée se trouve un amas galactique de masse M . La relativité générale prédit la
déviation des rayons lumineux à proximité de ce corps massif : c'est le
phénomène de lentille gravitationnelle. On peut modéliser cet effet par une
lentille convergente dont la distance focale dépend de la position du rayon
lumineux suivant la relation $\delta\phi=\frac{4GM}{c^2r_0}$, où $r_0$ est la distance
du rayon lumineux au centre de la lentille et $G$ la constante de gravitation.
\subQuestion Déterminez la distance focale associée au rayon qui frappe la lentille à une distance r 0 du centre
optique.
\subQuestion On suppose la lentille à une distance D de la Terre. Déterminez le
paramètre $r_0$ des rayons qui sont effectivement observés depuis la planète.
\subQuestion En déduire l'ouverture angulaire sous lequel l'objet est vu. Pour
une source ponctuelle située à l'infini, expliquez pourquoi on observe un anneau
(dit anneau d'Einstein).
\subQuestion On mesure un anneau d'ouverture angulaire $\alpha = 2.10^{-6}$ amplifié par
une lentille distante de $D = 10^9 al$ . Déterminez la masse de la lentille. On
prendra $1al \simeq 9.10^{15} m$.
\end{Exercise}
\begin{Answer}
  $\delta\phi = tan(\delta\phi)  =r_0/f' \implies f' = \frac{c^2r_0^2}{4GM}$ Les rayons viennent
  de l'infini donc $D = f' = \frac{c^2r_0^2}{4GM} \implies r_0 = \pm 2
  \frac{\sqrt{GMD}}{c}$. On voit donc un anneau car seul ces deux $r_0$ sont
  visibles + symétrie rotation. L'ouverture angulaire est alors: $\frac{r_0}{D}= \frac{2}{c}\sqrt{\frac{GM}{D}}$.

\end{Answer}
