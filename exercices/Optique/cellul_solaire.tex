\begin{Exercise}[title=Enduit pour cellules solaires]
  Une cellule solaire est fabriquée en silicium $(n=3,5)$ et surmontée d'un film
  transparent en $SiO(n=1,45)$. Quel est le rôle du film ? Quelle doit être son
  épaisseur ? Par beau temps, le soleil nous procure en surface de la Terre
  \SI{1}{kW/m^2}. Le rendement typique de ce type de cellule est de 20 \%. Calculer la
  surface des panneaux solaires équivalent à une installation électrique
  standard $(230V - 30A)$
\end{Exercise}
\begin{Answer}
  Le film est là pour limiter les reflets et améliorer le rendement de la cellule.
  Le système est équivalent à une interférence dans une lame fine on peut calculer
  la différence de marche (arrivée des rayons normalement au dioptre) : $\delta = 2n_{film}e_{film}$ . On se place dans le cadre d'une différence de marche destructive
  pour limiter la réflexion et on prends l'ordre le plus petit pour limiter les
  dissipations dans la couche $e_{film} = \lambda_0/4n_{film} = 95 nm$.
  $P = UI$ il faut donc \SI{34,5}{\m^2} de cellule.
\end{Answer}
