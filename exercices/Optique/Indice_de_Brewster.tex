\begin{Exercise}[title=Indice de Brewster]
Un rayon lumineux arrive à l’interface plane séparant l’air d’un milieu d’indice n. Il se scinde
en un rayon réfléchi et un rayon réfracté. On souhaite obtenir que ces deux rayons soient
perpendiculaires. Déterminer la valeur des différents angles.
Application numérique : $n = 1, 33.$
\end{Exercise}
\begin{Answer}
{brewster}
On note i l’angle d’incidence, $i_1$ l’angle de réflexion et $i 2$ l’angle de réfraction. Les lois de Descartes s’écrivent $i = - i_1$ et $n \sin i = n'\sin i_2 $ et on cherche $i_2-i_1 = \frac{\pi}{2}$. En combinant les relations, on obtient $n\sin(-i_1 ) = n'\sin i 2$ avec $-i_1 = \frac{\pi}{2}-i_2$ soit $n \cos i_2 = n' \sin i_2$ . On a donc $\tan i_2 = \frac{n}{n'}$ .
Application numérique : $n = 1, 00$ et $n' = 1, 33$ soit $i_2 = 36^o 56'$.
\end{Answer}
