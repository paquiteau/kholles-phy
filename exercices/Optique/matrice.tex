\begin{Exercise}[title=Traitement Matriciel]
   On considère un rayon lumineux arrivant sur une lentille convergente de
   distance focale $f_0$ . A une distance $l_1$ du foyer objet, le rayon est à
   une distance $x_1$ de l'axe optique et forme un angle $\alpha_1$ avec ce dernier.
   A une distance $l_2$ du foyer image, le rayon est à une distance $x_2$ de
   l'axe optique et forme un angle $\alpha_2$ avec ce dernier.
   \begin{center}
     \begin{tikzpicture}
       \draw[->] (-5,0) -- (0,0) node[below left]{$O$} --(5,0);
       \draw[<->, very thick] (0,-2) --(0,2);
       \coordinate (A) at (-4,1.8);
       \coordinate (B) at (4.2,-2);
       \coordinate (C) at (0,1);
       \coordinate (F) at (-2.5,0);
       \coordinate (F1) at (2.5,0);
       \node  at (F){|};
       \node  at (F1){|};
       \node[above=0.1cm] at (F){$F$};
       \node[above=0.1cm] at (F1){$F'$};
       \draw (A) ++(150:1) -- (C) -- (B) -- ++(-37:1);
       \draw[dashed] (A) ++(-1,0) -- (0,1.8)  (B)++(1,0) -- (0,-2);
       \draw[<-] (A) -- (-4,0) node[midway, left]{$x_1$} -- (F) node[midway, below]{$l_1$}
       (B) |- (4.2,0) node[near start,right]{$x_2$}-- (F1) node[midway, below]{$l_2$}
       (C) -- (0,0)node[midway,left]{$x_1'$}
       (2.5,-0.8) -- (F1) node[midway,right]{$x_2'$};
       \draw (A) arc (180:165:1) node[midway, left]{$\alpha_1$}
       (B) ++(0.5,0) arc (0:-37:0.5) node[midway, right]{$\alpha_2$};
       \end{tikzpicture}
   \end{center}

Question préliminaire : indiquez le signe des angles $\alpha_1,\alpha_2$
et des distances $x_1,x_2,x'_1,x'_2$.
\Question Exprimer en fonction de $x_1$ et $\alpha_1$ la distance $x_1'$ entre le centre optique et le rayon lumineux.
\Question Exprimer en fonction de $\alpha_1$ la distance $x_2'$ entre le foyer image
$F$ et le rayon lumineux.
\Question Exprimer $x_2'$ en fonction de $\alpha_2$ et de $x_1'$ En déduire
l'expression de $\alpha_2$ en fonction de $\alpha_1$ et $x_1$.
\Question Exprimer $x_2$ en fonction de $\alpha_1$ et $x_1$.
En déduire l'expression de $M$ tel que  $
\begin{pmatrix}
  x_2\\
f'\alpha_2
\end{pmatrix} = M
\begin{pmatrix}
  x_1\\
  f'\alpha_1
\end{pmatrix}
$

\end{Exercise}
\begin{Answer}

  En traçant le rayon parallèle passant par le centre optique :
  $\frac{x_1-x_1'}{l_1+f'}=-\alpha_1$ donc $x_1'=x_1+\alpha_1(l_1+f')$. De même
  $x_2'=\alpha_1f'$. Avec le rayon réel, $\frac{x_1'-x_2'}{f'}=-\alpha_2$ donc
  $\alpha_2=-\frac{x_1}{f'}- \frac{\alpha_1l_1}{f'}$. De même :
  $\frac{x_1'-x_2}{l_2+f'}=-\alpha_2$. Et on en déduit : $x_2 =-x_1
  \frac{l_2}{f'}+\alpha_1\left(f'-\frac{l_1l_2}{f'} \right)$ Donc:
  \[
M =
\begin{pmatrix}
  \frac{-l_2}{f'} & 1-\frac{l_1l_2}{f'^2}\\
  -1 & \frac{-l_1}{f'}
\end{pmatrix}
  \]

\end{Answer}
