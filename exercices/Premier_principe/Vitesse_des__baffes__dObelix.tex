\begin{Exercise}[title=Vitesse des « baffes » d’Obélix]
		Imaginez qu’Obélix vous gifle ! Vous ressentez une rougeur à la
		joue. La température de la région touchée a varié de \SI{1,8}{\degreeCelsius}.En supposant que la masse de la main qui vous atteint est de	1, 2 kg et que la masse de la peau rougie est de 150 g, estimez la vitesse de la main juste avant l’impact, en prenant comme	valeur de la capacité thermique massique de la peau de la joue : \gdr{c_joue}{3.8}{\kilo\J\per\kg}
\end{Exercise}
\begin{Answer}
		$v= \sqrt{\frac{2m_jc\Delta T}{m_m}}$ = \SI{4,4}{\m\s} = \SI{149}{\km\per\hour}
\end{Answer}
