\begin{Exercise}[title=Valeur en eau]
		On mélange 95g d'eau à \gdr{T_1}{20}{\degreeCelsius} et 71 g d'eau à \gdr{T_2}{50}{\degreeCelsius}.
		\Question Quelle est la température finale à l'équilibre ?
		\Question Expérimentalement on obtient \SI{31,3}{\degreeCelsius}. Expliquer.
		\Question en déduire la valeur en eau du calorimètre.
\end{Exercise}
\begin{Answer}
		Le système considéré est constitué du calorimètre et des deux masses d'eau qu'on y met.
		\Question En négligeant la capacité thermique du calorimètre on a $\Delta H_1 +\Delta H_1 = 0 $ ( pas de travail ni d'échange avec l'extérieur). $t_f = \frac{m_2\theta_2+m_1\theta_1}{m_1+m_2} =$ \SI{32,8}{\degreeCelsius}.
		\Question on ne peux pas négliger la capacité du calorimètre. de plus la température mesurée est inférieur à la température théorique , on en déduit que l'on a commencé avec l'eau la plus froide dans le calorimètre.
		\Question en notant $m_0$ la valeur en eau du calorimètre :
		\[ (m_0+m_1)c_0(\theta_f'-\theta_0) +m_2c_0(\theta_f'-\theta_2)=0  \implies m_0 = m_2\frac{\theta_2-\theta_f'}{\theta_f'-\theta_1}= 22,5 g\]
\end{Answer}
