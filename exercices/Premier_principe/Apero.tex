\begin{Exercise}[title=Apéro]
		Un glaçon flotte à la surface de l’eau dans un verre. Que peut-on en conclure quant à la masse volumique de l’eau solide et celle de l’eau liquide ? Lorsque le glaçon a fondu, le niveau de l’eau
		dans le verre est-il monté ? descendu ? resté inchangé ?
\end{Exercise}
\begin{Answer}
		Le glaçon flotte, $\rho_{glace} < \rho_{eau}$ . Quand le glaçon à fondu. Le niveau reste identique (90\% du volume du glacon est immergé, mais quand le glaçon fond il est remplacé par un  volume d'eau liquide faisant 90\% de son volume initial)
\end{Answer}
