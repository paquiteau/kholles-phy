\begin{Exercise}[title=Congélateur]
  Une machine frigorifique fonctionne réversiblement entre deux sources à \SI{0}{\degreeCelsius} et \SI{20}{\degreeCelsius}. La source chaude représente l’atmosphère et la source froide une salle parfaitement calorifugée dans laquelle est stockée de la glace qui est maintenue à \SI{0}{\degreeCelsius} grâce à la machine frigorifique.
  Calculer le prix de revient de 1 tonne de glace sachant que l’eau est introduite dans la salle frigorifique à la température de \SI{20}{\degreeCelsius}.
		\emph{Données : \gdr{c_p}{4,18}{\kJ\per\K\kg}, \gdr{L_{fus}}{330}{J\per\kg} 1kWh = 0.15\euro{} et $1m^3$ = 3\euro{}}
\end{Exercise}
\begin{Answer}
		On refroidit l'eau à \gdr{T}{0}{\degree} et on effectue le changement d'état (grandeurs massique) :
		$\Delta u = c_p \Delta T + L_{fus}$
		On a donc pour 1 tonne une énergie totale \gdr{E}{8.23e7}{J}. Donc on a un prix de (\gdr{kWh}{3.6e6}{J}) : \gdr{p}{25.86}{}\euro{}
\end{Answer}
