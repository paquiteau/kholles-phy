\begin{Exercise}[title=Condensateur de centrale à vapeur]
  Dans une centrale électrique de grande puissance, le condenseur est en charge
  de récupérer l’eau à la sortie des turbines et de lui retirer de l’énergie
  pour qu’elle puisse retourner à l’état liquide et ainsi ré-intégrer le circuit
  pompes $\to$ chaudières $\to$ turbines. L’eau du circuit
  (\SI{180}{\tonne\per\hour}) arrive à~\SI{0,5}{\bar} avec un volume spécifique
  de~\SI{3,1247}{\metre\cubed\per\kilogram} ; elle doit repartir à la même
  pression, à l’état de liquide saturé.

	Pour extraire de la chaleur à l’eau de la centrale, les condenseurs
    utilisent un circuit d’eau secondaire provenant directement d’une rivière.
    On y prélève de l’eau à~\SI{10}{\degreeCelsius}.

	Pour réduire l’impact écologique de la centrale, on souhaite rejeter l’eau
    secondaire dans la rivière à une température égale ou inférieure
    à~\SI{35}{\degreeCelsius}.

    \Question Quel débit d’eau secondaire doit-on prélever en rivière ?
    \Question Pour limiter les rejets de chaleur en rivière, où (et comment)
    rejette-t-on aussi, en pratique, la chaleur du condenseur ?

\end{Exercise}
\begin{Answer}
  \Question  $x_\A = \SI{96,44}{\percent}$ \& $x_\B = 0$ ; ainsi $\dot
  Q_\fromatob = \SI{-111,13}{\mega\watt}$ ce pourquoi il nous faut
  $\dot{m}_{\text{secondaire}}\ge$\SI{1062,4}{\kg\per\s}.
  \Question  Dans l’atmosphère, au moyen de l’eau secondaire rejetée par les larges

\end{Answer}
