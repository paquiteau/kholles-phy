\begin{Exercise}[title=Chauffage d'un gaz à l'aide une résistance]
		Soit un système piston-cylindre contenant V 1 = 0, 5 m 3 d’azote	à \gdr{P_1}{400}{kPa} et à \gdr{theta}{27}{\degreeCelsius}.
		l'élément chauffant électrique est allumé, et un courant $I = 2A$	y circule pendant $\tau = 5 min$ sous la tension $E = 120 V$.
		L’azote se détend de manière isobare. Au cours de cette transformation l’ensemble \{gaz, cylindre, élément chauffant \} cède à l’extérieur un transfert thermique Q ext = 2 800 J.
		Déterminer la température finale $T_2$ de l’azote.
		\emph{Donnée : \gdr{c_p}{1.039}{\kJ\per\K\kg}}.
\end{Exercise}
\begin{Answer}
		\gdr{T_2}{329,7}{K}.
\end{Answer}
